\chapter{引言}

\section{项目背景和意义}
% 1 pages
% 漏洞多
随着网络技术和应用的飞速发展,信息系统安全正面临着前所未有的挑战。网络化和互联互通性已经成为当前软件和信息系统发展的大势所趋,信息系统与互联网或其他网络的互连,将导致系统受攻击面增大,使系统面临的安全威胁空前的增加。另一方面,随着构建在信息系统之上的各种业务应用的不断丰富,软件和信息系统复杂程度的不断提高,系统中隐藏的各种安全隐患也越来越多,并且通常难以被发现和消除,尽管开发者已经投入了大量精力进行安全编程,但是软件漏洞仍然存在,并将继续成为一个重大的问题~\cite{vuldeepecker}。根据国际通用漏洞发布组织MITRE统计,在1999年,仅有1600个通用漏洞(CVE,Common Vulnerabilities and Exposures)被发现;到了2014年,新发现的CVE数量已接近10000个~\cite{liujian2018};而至今(2020年1月26日),该数字已经上升到了129695个~\footnote{\url{https://cve.mitre.org/cve/}}。同时,根据国家互联网应急中心(CNCERT/CC,National Computer Network Emergency Response Technical Team/Coordination Center of China)的报告~\footnote{\url{https://www.cert.org.cn/publish/main/upload/File/2019 First half year .pdf}}显示,仅2019年上半年,国家信息安全漏洞共享平台(CNVD)就收入了通用型安全漏洞5859个,其中Web漏洞占比24.9\%,位居第二。

% 危害大
从以往发生的安全事件来看,针对Web漏洞的攻击可导致的后果极为严重,一个网站系统只要有一处脆弱点,攻击者就可以使用一切能够使用的手段对该网站发起攻击,篡改网页内容,窃取服务器数据甚至在服务器中执行恶意代码,造成更严重的后果~\cite{WebApplication}。

% 静态扫描重要性
源代码是构成软件系统的基石。对于源代码进行安全扫描,可以在系统上线前及时发现各类安全漏洞和威胁,根据2019年中国软件评测中心的《2019大中型政企机构网络安全建设发展趋势研究报告》显示,进行源码扫描可以有效减少10\%-50\%的安全漏洞,降低软件安全建设和运维成本。因此,源码安全作为软件安全的重要一环,日益收到中大型政企机构的重视,在此方面的投入也不断提高。因此,构建一款具有高准确率的静态代码分析(SCA,static code analysis)系统就显得尤为重要。

% 目前静态扫描存在不足
随着软件数量和规模的扩大,人们总希望能够通过自动化技术发现代码中的漏洞,因此从开源工具到商业工具,从工程应用到学术研究,目前已有许多关于代码静态扫描系统的应用和研究。然而目前的代码安全扫描系统仍存在或多或少的问题,在工业界,SCA 使用了基于模式匹配、数据流分析等技术,往往希望不漏掉真实漏洞——不产生漏报,而产生了大量误报。过高的误报会大大加大人工审核的任务,而过高的漏报则可能导致 SCA 不可用。学术界对提高 SCA 的准确性进行了更深入的研究,引入了诸如符号执行,机器学习等技术,但是受限于计算能力,它们往往只能应对小规模代码项目,如何让学术界成果应用到工业界,使其适应大规模程序,是安全工程师亟待考虑的问题。

% 本文工作
本文旨在设计一款基于污点传播和机器学习的Java静态的代码扫描系统,填补了学术界和工业界的鸿沟,通过限制调用图的去依赖切片技术,将污染流分段处理,并将子污染流进行基于BLSTM模型的预测。具体来说,用户输入一个已编译好的jar包后,该系统以污点传播为基础,先产生一个初步的漏洞结果。由于污点传播存在过污染、无法识别清洁函数问题,该结果存在大量误报,接着系统根据初步结果进行污点传播分解,将其划分为多个子污染流,对每个污染流进行代码切片,并用BLSTM算法模型预测代码切片是否能够向下传播污点,判断整个污染流是否成立,并将疑似误报的漏洞威胁等级降低,以此让安全工程师或软件开发人员将注意力更多的放在更可能为漏洞的结果上。同时,用户可以对判断不准确的漏洞重新标注,通过模型周期性的迭代学习,其预测准确率将不断提高。

\section{研究现状}
本系统旨在利用传统方法配合机器学习对Java软件源代码进行静态安全扫描,因此本节将详细介绍工业界传统静态扫描系统应用现状和学术界基于机器学习的静态安全扫描研究现状。\\

\subsection{传统静态安全扫描应用现状}
% 1pages
C/C++语言的检测手段\\
Java语言的检测手段\\
Fortify\\

Coverity https://www.synopsys.com/software-integrity/security-testing/static-analysis-sast.html\\
checkmarx https://www.checkmarx.com/\\
Find-Sec-Bugs

其他厂商自研工具

\subsection{基于机器学习的静态安全扫描研究现状}
%1pages

\section{本文主要研究工作}


\section{本文组织结构}
asd