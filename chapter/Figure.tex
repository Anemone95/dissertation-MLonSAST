\chapter{图形}

\section{基本图形}
相对于表格而言,LaTeX中的图形就简单多了,需要注意的是本模板推荐将所有图形都转化为pdf,具体内容参见图~\ref{fig_errorExpCH4}。
该图形放在本模板的本地文件夹FIGs中。
图~\ref{fig_errorExpCH4}是将Excel的五个子图形排布在一个ppt页面上,之后保存为pdf文件,最终得到的图形可以保证是矢量图。


\begin{figure}[htb]
  \centering
  \includegraphics[width=5in]{FIGs/chapter4/errorExpCH4.pdf}
  \caption{以含错误的RTM为输入的五个系统上三个实验(Call,Data,Call+Data)的错误率(Incorrectness)}\label{fig_errorExpCH4}
\end{figure}

\textbf{注意:不要删除FIGs下面的njulogo和njuname这两个文件,这是论文封面的校徽和手写体南大校名。}

\section{引用代码}

\begin{figure}[htb]
  \centering
  \includegraphics[width=\linewidth]{FIGs/chapter4/VoDCodeSample.pdf}
  \caption{VoD系统中的代码片段}\label{fig_VoDCodeSample}
\end{figure}

这里给出一个代码引用的推荐实践。
引用代码时先将代码放入word的文本框中,调整结束后,将该文本框页面另存为pdf文件,之后再作为图形来引用,如图~\ref{fig_VoDCodeSample}所示。


