\chapter{总结与展望}
\section{总结}
为了解决应用安全问题,本文设计了一套基于污点分析、程序切片和 BLSTM 的静态安全扫描系统,旨在从源头处遏制程序漏洞。相较于传统的污点分析类代码扫描系统,本系统对污点传播树进一步分析,通过程序切片和先前标记进行学习,能够有效地排除误报,保证扫描结果准确性,大大降低了系统使用时的人力成本。本文主要工作如下:


\begin{enumerate}
    \item 本文对开源污点传播工具 Find Security Bugs 进行改造,使之能在报告中展示污点传播树,友好地向用户展示可能的漏洞利用过程。
    \item 系统对污点传播树进行拆分,将污点传播流和子传播流作为切片单位进行程序切片,系统优化了 Joana 切片过程,使之能够在缺失依赖、异构 Jar 包的情况下切片,并通过限制调用图限制程序切片范围,结合污点传播树的拆分,解决了目前切片资源消耗大导致扫描任务无法及时结束的问题。
    \item 系统根据实际情况对切片后的 SSA 进行泛化处理,进一步保证预测模型的泛化能力。
    \item 系统通过对漏洞的各个污点传播树的传播流及其子污染流进行预测,从而预测漏洞是否为误报。系统采用 BLSTM 模型,该模型在学术界已被多项工作证明在漏洞预测领域有较为显著的效果。用户在系统界面中能够得知漏洞预测结果,若为误报,系统能够清晰给出判断依据,即污点在哪一段传播流中无法继续传播,一定程度上弥补了深度学习可解释性差的问题。安全工程师能够轻松地对漏洞、传播树和传播流进行标记,并发起新的学习任务,不断提高预测模型的准确率。
\end{enumerate}

系统被设计为有良好的可拓展性,对于污点分析结果翻译、切片器和预测器,系统都对其高度抽象为接口,方便后期根据实际需要对系统中各个模块所用技术进行更换。

经过测试和实验,结果表明本系统具有较高的健壮性,在 Maven 仓库中前 100 流行度的项目中系统扫描成功率为 100 \%,此外,本系统具有较高准确性,在 OWASP 数据集中,本系统预测模块能够准确预测安全污点传播流,相较于 Find Security Bugs,本系统在 OWASP 数据集上提高了约 20 \% 的准确率,减少了 25.44 \% 的误报。

\section{展望}

本系统是将深度学习应用于漏洞扫描领域的一次成功探索,实现了传统检测技术与深度学习相结合,对 Java 代码进行更准确的安全扫描任务,然而在未来,系统在以下方面还有较大的改进空间:

\begin{enumerate}
    \item 系统目前只针对 Java 语言,在未来可以将其方法推广到其他语言的项目。
    \item 系统只能适用于能用污点分析方法分析的漏洞类型,在未来可以将切片和预测方法推广到其他基础分析的误报排除中。
    \item 系统并不能发现更多漏报,在未来可以使用类似的切片技术和预测方法,结合其他经典漏洞挖掘技术,在消除漏报方面做进一步提升。
    \item 系统在特征表示时,目前实际上是将切片转化为单词序列,再将单词序列进行向量化处理,在未来可以参考程序图特征表示的前沿工作,将切片表示成信息更丰富的特征,进一步提高预测准确性。
    \item 目前由于数据量有限,对于所有污点漏洞均使用一个模型进行预测,待数据量进一步扩大时,可以将针对每一种漏洞单独训练模型。
\end{enumerate}