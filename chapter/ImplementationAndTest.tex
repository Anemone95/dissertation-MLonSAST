\chapter{Java静态安全扫描工具实现和测试}
\section{污点分析模块的实现}
\subsection{构造污点传播函数调用图}
%数据结构

%构造方法

\subsection{构造污点传播树}
%数据结构
%构造方法
%数据保存形式

\section{程序切片模块的实现}

\subsection{后向切片的实现}
% 先说流程

% joana配置
%突出解决了的几个坑(bug,无依赖切片,限制调用图的切片)

% 处理的bug,SpringBoot类的包找不到类: 1. top.anemone.mlsast.core.classloader.AppWalaClassLoaderFactory top.anemone.mlsast.core.classloader.AppWalaClassLoader 3.top.anemone.mlsast.core.classloader.AppClassloader

% 解决无依赖切片
% 1.makewithroot 当A类继承B类,找不到B类时,A类将继承Object,
% 2. top.anemone.mlsast.core.joana.AppEntrypoint#makeParameterTypes,即入口函数参数为缺失依赖时直接返回一个类型标记而不继续找
% 3. top.anemone.mlsast.core.joana.AppEntrypoint#makeArgument,即遍历函数的参数为到找不到的类时,argument会新建一个空的虚节点代替

% 对限制调用图的切片 edu.kit.joana.wala.core.prune.NodeLimitPruner


\subsection{切片控制模块实现}

% 对污点传播结果的反序列化和翻译

% 对污点传播结果的dfs算法 top.anemone.mlsast.core.slice.DFSTaintTree

% 对每一个污染流进行切片代码


\section{数据处理模块的实现}

\subsection{泛化处理的实现}

\subsection{向量化处理的实现}


\section{误报预测模块的实现}

\subsection{误报预测控制}
%突出流程图

\subsection{误报预测时序图}

\subsection{误报标记时序图}

\section{系统测试与运行展示}
\subsection{测试目标}
\subsection{功能测试}
\subsection{性能测试}
选择数据量较大的jar包进行测试
\subsection{系统运行展示}
\subsection{系统效果评估}