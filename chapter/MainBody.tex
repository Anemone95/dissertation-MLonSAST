\chapter{正文}

\section{正文书写的小技巧}
CTeX自带的pdf浏览器,双击每段文字之后会自动回到WinEdt的编辑位置。

只有间隔一个明显的换行才会自然段分段。

因此,建议把一个自然段中的每句话都单独作为一行。
这样的好处是,每次双击一句话,都可以回到WinEdt编辑器具体的一行。
如果编辑时也按照自然段组织,则双击时会返回到一大段,不能定位到具体位置。
不便于快速定位到出现问题的地方。

\section{一些正文中的标记}
\emph{斜体}

\textbf{加粗}

\texttt{代码元素格式}

\begin{center}
居中,左右对齐同理。
\end{center}

这里展示脚注。\footnote{数字列举和圆点列举见摘要部分}

一个小建议,中文后直接跟上述格式标记(包含各种引用)可能会出现一些问题。
因此,在中文字和格式标记的斜杠之间加入~\emph{一个波浪号}是一个常用的习惯。
双~~波~~浪~~线等价于一个强制空格,有时比键盘输入的空格要好用。


\section{注意软换行的使用}
论文一般会引用代码,本模板建议将代码声明为~\texttt{class.this()}格式。
在引用代码时,较长的函数名会导致函数名超出文本边界的情况,因此可以考虑手动进行软换行,请参考以下例子。

“图XX 展示了从AquaLush 系统中抽取的函数调用依赖示例,其中~\texttt{UICon-} \linebreak \texttt{troller.buildLogScrn()} 是为了实现新功能“the control panel shows log message”而在新版本中添加的函数。”
