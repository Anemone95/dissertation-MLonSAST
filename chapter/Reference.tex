\chapter{论文引用}

此处的论文引用采用的是类似于IEEE的按出现位置的数字编号格式。建议将被引用的论文全名放入dblp网站(必应谷歌搜索dblp)搜索,之后进入该论文详细信息,如图~\ref{fig_dblpForBibtexCH7} 所示。

\begin{figure}[htb]
  \centering
  \includegraphics[width=5in]{FIGs/chapter7/dblpForBibtex.pdf}
  \caption{在dblp上下载Bibtex}\label{fig_dblpForBibtexCH7}
\end{figure}

点击该链接之后将得到Bibtex信息,如图~\ref{fig_bibtexDetailCH7}所示。
打开本地文件夹下的reference.bib文件,完整添加该信息。
并在需要引用的位置添加这一引用~\cite{DBLP:journals/computer/EgyedZHD18}。
格式为bibtex信息中的开头,\emph{例如图中的“DBLP:journals/computer/EgyedZHD18”。(此处是一个典型的因为长字符串导致的bad box,请参考上述章节的内容手动完整软换行)}。

\textbf{注意:在修改并保存reference.bib文件后,先点击PDFLaTeX旁边的B按钮编译bib文件,之后需要连续使用PDFLaTeX编译两次,直到最后控制台输出的Warnings不再增加,此时才完成一次论文引用的更新。}

在bib文件中出现,但并未在论文中被引用的论文不会出现在最后的参考文献中。如果dblp中并未包含你需要的论文,则可以尝试谷歌或百度学术的搜索结果,一般也包含bibtex信息,但可能不完整或不规范。

引用网站链接可以考虑这一格式~\cite{GanttSystem}(不推荐,网站链接使用脚注更规范些)。

引用书籍可以考虑这一格式~\cite{Pohl2010Requirements}。

中文文献请参考这一格式~\cite{cyg2006}(引用标记请避免中文,否则容易出现编译错误)。

以下英文引用用来测试引文排序是否按照插入顺序,以及多引文是否合并~\cite{DBLP:journals/computer/EgyedZHD18, DBLP:journals/ml/TingZCZWZ19}

\begin{figure}[htb]
  \centering
  \includegraphics[width=5in]{FIGs/chapter7/bibtexDetail.pdf}
  \caption{Bibtex详细信息}\label{fig_bibtexDetailCH7}
\end{figure}

