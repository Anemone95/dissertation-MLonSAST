\chapter{相关技术综述}
这是章节标题。
注:一般而言,标题不要比小节标题更小,即不要出现1.2.3.4这种标题(本模板支持此类标题,即Subsubsection)。

\section{漏洞挖掘技术}
漏洞挖掘指用自动化或半自动化技术对软件进行本身进行静态,动态分析,检测其是否存在安全漏洞的过程~\cite{刘剑2018软件与网络安全研究综述}。随着软件规模扩大,软件功能种类多样化,安全漏洞其种类也在不断增多,不同漏洞的产生原因不同,利用方式也不同,因此,往往一种漏洞挖掘技术不能适用于所有漏洞,本章就目前常用的漏洞挖掘技术进行分类并分别进行简要介绍~\cite{刘剑2018软件与网络安全研究综述,梅宏2009软件分析技术进展}。\\

% 需要写java的常见漏洞吗?
\subsection{基于代码分析的漏洞挖掘技术}
\subsubsection{词法分析技术}

\subsubsection{数据流分析技术}

\subsubsection{形式化方法分析技术}
形式化方法分析主要思想是将软件代码性质进行形式化描述,再判断该描述是否满足漏洞特征,其中定理证明技术是形式化代码分析技术的主要代表。

%https://firmianay.gitbooks.io/ctf-all-in-one/doc/5.0_vulnerability.html
定理证明技术将源程序代码特征表述为一组数学条件,接着再定义一组逻辑推导公式,漏洞挖掘过程类似于数学上的定理证明过程。\\

\subsubsection{符号执行技术}
% https://www.youtube.com/watch?v=mffhPgsl8Ws
% https://blog.csdn.net/wcventure/article/details/86773290
%Symbolic Execution for Software Testing: Three Decades Later
该技术将程序中变量

\subsection{基于模糊测试的漏洞挖掘技术}

\subsection{基于机器学习的漏洞挖掘技术}

\section{污点传播算法}

\subsection{污点传播原理介绍}

\subsection{污点传播的优势}

\subsection{污点传播的不足}

\section{程序切片技术}
\subsection{程序切片技术介绍}

\subsection{后向程序切片的优势}

\section{词嵌入技术}
(后期可能会改为文档嵌入技术,先不写)

\section{LSTM算法}
\subsection{LSTM原理介绍}
\subsection{LSTM的优势}

\section{本章小结}