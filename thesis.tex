%%]dvipdfm
\expandafter\def\csname CTEX@spaceChar\endcsname{\hspace{1em}}
\documentclass[master]{NJUthesis}
%\documentclass[twoside, master]{NJUthesis}
% 可选参数:
%   review 审阅模式,激活后个人、导师与学校信息均被隐去
%   oneside/twoside 单面/双面打印
%   phd/master 博士/硕士论文
% 下面三个选一个:
% dvi2pdf 使用 dvi2pdf(x) 生成最终的 PDF 文档 (缺省设置,不建议修改)
% dvips 使用 dvips 生成最终的 PS 文档
% pdftex 使用 pdfLaTeX 生成最终的 PDF 文档

%%%%%%%%%%%%%%%%%%%%%%%%%%%%%%
%% 导言区
%%%%%%%%%%%%%%%%%%%%%%%%%%%%%%

% 小节标题靠左对齐
\CTEXsetup[format+={\flushleft}]{section}

% 设置链接颜色
\hypersetup{
% pdf 属性
            pdftitle={LaTeX Thesis Template of Nanjing University}, %
            pdfauthor={San Zhang}
}

% 表格
\usepackage{longtable, multirow}
% 英文使用 Times 字体
\usepackage{times}
% 源代码
\usepackage{fancyvrb}
% 自定义列表样式
\usepackage{enumitem}
\usepackage{url}
\usepackage{amsmath}
\usepackage{amssymb}
\usepackage{moreverb}
\usepackage{txfonts}
\usepackage{mathcomp}
\usepackage{graphicx}
\usepackage{subfigure}
\usepackage[linesnumbered,boxed,ruled,vlined]{algorithm2e}
\usepackage{array}
\usepackage{multirow}

%%	added by Jiang
\usepackage{extarrows}	%使用长箭头
\usepackage{nomencl}	%与术语表有关的包
\usepackage{booktabs}
\usepackage{ccmap}

%% added by lhy
%%取消默认楷书命令
\let\kaishu\relax 
%% 配置新的楷书命令,粗体用方正粗楷简体,普通用方正楷体简体
%% 这里其实是可选的,如果有什么更合适的楷体字体,可以自行替换
\setCJKfamilyfont { bfkt }[BoldFont=FZCKJT.ttf]{FZKTJT.ttf}
%% 添加新的字体命令kaishu,中文用方正楷体,英文用times
\NewDocumentCommand \kaishu { } { \CJKfamily { bfkt } \fontspec{Times New Roman}}
%% 全文所有英文默认使用Times英文字体
\setmainfont{Times New Roman}

\makenomenclature

\setcounter{topnumber}{5}

\theoremstyle{plain}
\newtheorem{definition}{\hspace{2em}定义}[chapter]
\newtheorem{lemma}{\hspace{2em}引理}[chapter]
\newtheorem{theorem}{\hspace{2em}定理}[chapter]
\newtheorem{property}{\hspace{2em}性质}[chapter]
\newtheorem{example}{\hspace{2em}例}[chapter]
\newtheorem{myrule}{\hspace{2em}规则}[chapter]


\newcommand{\tabincell}[2]{\begin{tabular}{@{}#1@{}}#2\end{tabular}}% 表格内换行
\renewcommand{\footnoterule}{%脚注线
  \kern -3pt
  \hrule width 2.3in height 0.4pt
  \kern 2pt
}


\begin{document}

%%%%%%%%%%%%%%%%%%%%%%%%%%%%%%
%% 封面部分
%%%%%%%%%%%%%%%%%%%%%%%%%%%%%%

% 国家图书馆封面内容字符串
% 仅博士需要填写并保证模板参数选择了 phd
\classification{}
\confidential{}
\UDC{}
\titlelinea{南京大学学位论文}
\titlelineb{~\LaTeX{}~模板}
\titlelinec{}
\advisorinfo{南京大学~软件学院}
\chairman{XXX 教授}
\reviewera{某某某某 副研究员}
\reviewerb{XXX 教授}
\reviewerc{XXX 教授}
\reviewerd{XXX 教授}
\nlcfootdate{~年~~月~~日}

% 南大中文封面内容字符串
\title{基于污点传播和机器学习的Java静态扫描系统的设计和实现}
\author{徐文远}
\studentnum{MF1832200}
\grade{2018}
\advisor{\kaishu 陈振宇~~教授}

\major{\kaishu 工程硕士(软件工程领域)}
\researchfield{\kaishu 软件工程}
\footdate{\kaishu 20xx~年~x~月}
\submitdate{\kaishu 20xx~年~x~月~xx~日}
\defenddate{\kaishu 20xx~年~x~月~xx~日}



% 英文封面内容字符串
\englishtitle{Using a Hammer to Crack a Nut}
\englishauthor{Wenyuan Xu}
\englishadvisor{Professor }
\englishadvisorname{Zhenyu Chen}
\englishinstitute{Software Institute}
\englishdegree{Master of Engineering}
\englishmajor{Software Engineering}
\englishdate{May 2020}

% 制作封面命令
\maketitle

%\makechinesetitle

% 制作英文封面命令
\makeenglishtitle


%%%%%%%%%%%%%%%%%%%%%%%%%%%%%%
%% 前言部分
%%%%%%%%%%%%%%%%%%%%%%%%%%%%%%
\frontmatter

\begin{abstract}

这部分是中文摘要。

\textbf{注意:本模板使用的是PDFLaTeX编译的,这一编译的好处在于速度快,并能直接引用pdf格式的图形。}

以下展示列举(无编号):
\begin{enumerate}
  \item 贡献1。
  \item 贡献2。
  \item 贡献3。
\end{enumerate}

%这是注释,不影响正文

\keywords{中文,关键,字}

\end{abstract}

% 英文摘要
\begin{englishabstract}
This is English abstract.

以下展示的是圆点列举(无编号)做些修改~:
\begin{itemize}
  \item First Contribution.
  \item Second Contribution.
  \item Third Contribution.
\end{itemize}

\englishkeywords{English, Keywords}
\end{englishabstract}

% 生成目录命令(目录中不包含目录本身)
\addtocontents{toc}{\protect\setcounter{tocdepth}{-1}}
\tableofcontents
\addtocontents{toc}{\protect\setcounter{tocdepth}{3}}


% 以下两个目录可根据具体情况注释掉(将表格目录和图形目录重命名后加入目录)
% 生成表格目录命令
\renewcommand*{\listtablename}{表~~目~~录}
\listoftables
\addcontentsline{toc}{chapter}{表~~目~~录}
% 生成插图目录命令
\renewcommand*{\listtablename}{图~~目~~录}
\listoffigures
\addcontentsline{toc}{chapter}{图~~目~~录}

%生成术语表命令
%\include{chapter/Nomenclatures}
%\def\nomname{缩略语对照表}
%\printnomenclature[5em]

%%%%%%%%%%%%%%%%%%%%%%%%%%%%%%
%% 正文部分
%%%%%%%%%%%%%%%%%%%%%%%%%%%%%%
\mainmatter

\chapter{引言}

这是章节标题。
注:一般而言,标题不要比小节标题更小,即不要出现1.2.3.4这种标题(本模板支持此类标题,即Subsubsection)。

\section{项目背景和意义}

\section{研究现状}
\subsection{传统静态安全扫描应用现状}
\subsection{基于机器学习的静态安全扫描研究现状}

\section{本文主要研究工作}


\section{本文组织结构}

\chapter{相关技术综述}
本章主要介绍时下常用的漏洞挖掘技术,以及本系统将要用到的技术、工具和开发框架。

\section{漏洞挖掘技术}
漏洞挖掘指用自动化或半自动化技术对软件进行本身进行静态、动态分析,检测其是否存在安全漏洞的过程~\cite{liujian2018}。随着软件规模扩大,软件功能种类多样化,安全漏洞其种类也在不断增多,不同漏洞的产生原因不同,利用方式也不同,一种漏洞挖掘技术不能适用于所有漏洞,因此在实践中,人们会在软件开发的各个阶段应用不同类型的技术,本文将这些技术的应用场景主要分为三类:
\begin{itemize}
    \item  白盒测试:也称为静态测试,通常发生在软件编码阶段,对应用程序的源代码或对源码编译产生的二进制文件进行安全性审计,从而发现漏洞。该场景下的技术由于获取了源代码信息,也被称为源码扫描或静态扫描,其可以获得较高的代码覆盖率,发现更多的漏洞,但由于无法运行而产生大量误报,它也是本文系统的应用场景。
    \item 黑盒测试:通常发生在软件测试和运行阶段,对应用程序动态运行,并输入数据,分析程序反馈从而发现漏洞。适用于该场景的技术可以在程序只暴露接口的情况下展开测试,因此应用广泛,同时动态运行程序使其具有更低的误报,甚至产生漏洞利用报告,但由于如今应用程序复杂,自动化测试技术往往无法覆盖所有的代码逻辑。
    \item 灰盒测试:介于白盒和黑盒之间的一种测试,这一场景下的技术往往通过软件插桩或是逆向工程,不但关注程序输入输出信息,还可以了解部分程序内部逻辑。因此具备上述两种测试的优点,许多黑白盒测试技术稍加改造也可以变为灰盒测试技术,但由于其终究还是需要获取程序本身,以及需要动态运行,使用比纯黑百盒测试更为复杂。
\end{itemize}

在漏洞挖掘技术发展早期,每一种技术往往只能应用于一种特定场景,但随着研究者的不断完善和改进,如今一种技术也可以适用于多种场景,并且技术本身也产生了相互组合,对其正交分类较为困难,本章就目前常用的漏洞挖掘技术进行粗略分类并分别进行简要介绍~\cite{liujian2018,meihong2009}。


% 需要写java的常见漏洞吗?
\subsection{基于代码分析的漏洞挖掘技术}
这一类漏洞挖掘技术侧重于对程序代码本身进行分析,同时对漏洞产生原理进行建模,将程序分析结果结合漏洞模型发掘漏洞,主要用于白盒测试场景。主要有词法分析技术、数据流分析技术、形式化分析技术和符号执行技术。\\
\vspace{1cm}
\subsubsection{词法分析技术}
词法分析技术是最简单的一类漏洞挖掘技术,其主要思想是将代码文本与归纳好的缺陷模式进行匹配,以此发现漏洞。由于其不深入分析程序结构和语义,往往只能挖掘较为简单的一类漏洞,并且存在相当高的误报率,在实际场景下应用较少,但由于其思想简单,适用性很广,目前也还存在类似工具,如:MobSF~\footnote{\url{https://github.com/MobSF/Mobile-Security-Framework-MobSF}},Cobra~\footnote{\url{https://github.com/WhaleShark-Team/cobra}}。\\

\subsubsection{数据流和控制流分析技术}
数据流分析是一种按程序执行路径模拟数据流动的一种分析技术,其原本用于进行程序优化~\cite{Kildall1973},安全研究者们发现后将其运用于漏洞挖掘中,如今该技术在白盒,灰盒和黑盒测试都有应用~\cite{Shastry2016}。

在数据流分析过程中,存在过程内分析和过程间分析,过程内分析主要对函数内分析,而过程间的分析主要处理跨函数分析。
对于过程内分析,根据其对程序路径的分析精度,可分为流不敏感分析,流敏感分析和路径敏感分析。流不敏感的数据流分析只是按代码行号从上而下进行分析;流敏感分析会首先产生程序的控制流图(CFG, Control Flow Graph),再按照CFG的拓扑排序正向或逆向的分析;路径敏感信息不仅考虑到语句的先后顺序,还会考虑语句的可达性,即会沿实际可执行到路径进行分析。
过程间分析首先构造程序的调用图(CG, Call Graph),接着遍历图中的函数进行过程内分析,当遇到其他函数时,若已分析过,则直接使用分析结果向下分析,若未分析过,则跟进该函数,再次进行过程内分析,并且将分析结果保存。

数据流分析能够一定程度上理解程序语义,是一种比词法分析技术更为精确的一类分析技术,其关键在于准确的计算程序的数据流,此外,本文使用的污点分析技术作为数据流分析的一种特例,作为本系统所使用的技术之一,将在下文单独一章进行介绍。\\

\subsubsection{形式化方法分析技术}
形式化方法分析主要思想是将软件代码性质进行形式化描述,再判断该描述是否满足漏洞特征的一类分析方法~\cite{B:automatedTheoremProving},其中定理证明技术是形式化代码分析技术的主要代表。

%https://firmianay.gitbooks.io/ctf-all-in-one/doc/5.0_vulnerability.html
%Automated Theorem Proving in Software Engineering
%https://github.com/leanprover/lean2
定理证明技术将漏洞存在(或不存在)定义为一定理,再将源程序代码特征转化为数学表达形式,最后对数学表达进行逻辑推理,若定理存在性得以证明,则漏洞存在(或不存在),即漏洞挖掘过程类似于数学上的定理证明过程。主要代表性工具有 infer~\footnote{\url{https://fbinfer.com/}}~\cite{atp:infer}、 ESC/Java~\cite{atp:escjava} 和 saturn~\cite{atp:saturn}。

该技术作为一种使用严格的数理逻辑推理作为检测手段的技术,具有极低的误报率,但由于其需要针对特定漏洞构建数学条件,需要大量的人工参与,有的漏洞甚至难以用数学结构表达,导致其适用于死循环、资源泄露和空指针等问题,对新漏洞的扩展性不高,同时,如何将大规模程序应用于形式化方法分析也成为工业界亟待解决的问题。 \\

\subsubsection{符号执行技术}
符号执行技术是一种将程序执行可达性问题转化为约束求解问题,并以此进行漏洞挖掘的技术~\cite{sym:sum},代表性工具有angr~\footnote{\url{http://angr.io/}},DART~\cite{sym:dart}, CUTE~\cite{sym:cute}, EXE~\cite{sym:exe}和KLEE~\cite{sym:klee}。
% https://www.youtube.com/watch?v=mffhPgsl8Ws
% https://blog.csdn.net/wcventure/article/details/86773290
%Symbolic Execution for Software Testing: Three Decades Later

具体来说,符号执行包含一个符号状态表$\sigma$和一个符号路径约束$PC$,开始时,$\sigma=\varnothing, PC=true$,每读取一条语句,就将变量抽象为约束求解中的变量、常量或他们的表达式放入$\sigma$中,特别的,当遇到条件判断$if(e)$时,将if分支的$PC$更新为$PC \wedge \sigma(e)$,将else分支的$PC'$更新为$PC\wedge \neg\sigma(e)$,随后使用约束求解器求解$PC$和$PC'$,如果约束不满足,则停止对该分支的解析(因为该分支不可达)。当符号执行遇到程序崩溃、预先定义的漏洞语句、或是程序正常退出时,整个分析停止,同时可以计算可以到达停止点的输入。

符号执行可以分析程序中的控制流、覆盖更多的代码,同时也有效降低了误报率,但传统符号执行严重依赖于约束求解器的能力,例如,若约束求解器不能处理非线性计算,或是整个程序中存在无法分析的第三方库,那么整个分析将无法继续。为解决这些问题,研究者们提出了动态符号执行的想法~\cite{sym:dart,sym:cute,sym:exe,sym:klee},但其在实际应用中仍不是很广泛,主要原因在于其需要大量计算资源,甚至在处理大规模程序时,出现的路径爆炸问题会导致约束求解无法产生结果。\\

\subsection{基于模糊测试的漏洞挖掘技术}
模糊测试是一种通过构造大量非预期输入,同时观察软件运行反馈来发现软件漏洞的方法~\cite{fuzzingstateofart}。

由于其不需要了解程序内部具体实现,不论是Web应用还是二进制程序,其都是一种非常受欢迎的技术。
该技术的关键在于如何构造能够引发软件漏洞的输入,对于Web应用来说,扫描器会针对每个漏洞(如SQL注入,XSS等)准备若干个(或若干组)可能会引发漏洞的输入模式,接着爬虫程序会爬取网站所有URL(或是将URL也作为模糊输入的一部分),将输入模式整合进HTTP报文中并发送给服务器,若服务器返回符合漏洞特征(也被称为测试断言,Test Oracle),则报告程序存在漏洞,主要的工具有AWVS~\footnote{\url{https://www.acunetix.com/vulnerability-scanner/}},Netsparker~\footnote{\url{https://www.netsparker.com/}}和ZAP~\footnote{\url{https://www.zaproxy.org/}}。

学术界更热衷于对二进制程序的模糊测试技术进行改进~\cite{artoffuzz},为了构造能够到达更深层代码的输入,研究者们提出了基于变异的模糊测试和基于生成的模糊测试~\cite{Zou2018},基于变异的模糊测试通过当前模糊测试结果反馈和结合程序特征,对输入进行各类变异,以此指导测试方向,该方向主要有:代码覆盖率制导的模糊测试技术,如 AFL~\footnote{\url{http://lcamtuf.coredump.cx/afl/}}和libFuzzer~\footnote{\url{https://llvm.org/docs/LibFuzzer.html}};由符号执行制导的模糊测试技术,如 Driller~\cite{Driller}和由信息流制导的模糊测试技术,如 VUzzer~\cite{VUzzer}。基于生成的模糊测试技术适用于输入具有一定模式的场景,例如 PDF 阅读器,程序编译器或解析器等,例如 CodeAlchemist~\cite{CodeAlchemist} 即设计了一套 JavaScript 代码生成工具,以此发现 JavaScript 解释器的漏洞。

模糊测试作为一种能够得到漏洞利用输入的漏洞挖掘技术,在黑盒和灰盒场景下应用广泛,但由于目前程序日益复杂,该漏洞挖掘技术很难测试隐藏在复杂状态已经条件分支下的代码块,导致程序覆盖率不高,即其具有低误报,高漏报的特点。

\section{污点分析}
%https://www.k0rz3n.com/2019/03/01/%E7%AE%80%E5%8D%95%E7%90%86%E8%A7%A3%E6%B1%A1%E7%82%B9%E5%88%86%E6%9E%90%E6%8A%80%E6%9C%AF/
%https://github.com/firmianay/CTF-All-In-One/blob/master/doc/5.5_taint_analysis.md
%http://www.jos.org.cn/html/2017/4/5190.htm
%http://www.jos.org.cn/html/2019/2/5581.htm
污点分析属于数据流分析的变种,通过判断关键操作的数据(如调用危险函数的参数)是否可被用户操控,推测程序是否存在 安全性漏洞~\cite{taint:wanglei}。由于其了解程序上下文,并且有较强的可解释性——安全工程师可以通过跟踪污点传播过程判断是否存在安全问题,因此其也成为了挖掘 Web 或 Android 漏洞较为常用的技术,也被很多开源或商用白盒扫描器使用,如:Pixy~\cite{pixy}、Find Security Bugs ,Fortify~\footnote{\url{https://www.microfocus.com/en-us/solutions/application-security}}和LGTM~\footnote{\url{https://lgtm.com/}}。\\

\subsection{污点分析原理}

\subsubsection{污点分析三要素}
污点分析主要有三个组成要素:污点信息的产生点(source)、污点信息的汇聚点(sink)和污点信息的清洁点(clean),它们通常需要富有经验的安全工程师手动设置。

\begin{itemize}
	\item 产生点(source):污点的产生点往往是用户输入的数据,比如Web应用中读取URL参数的函数,顾名思义,这些函数调用后的返回值被标记为污点——攻击者可以操控的数据点。
	\item 汇聚点(sink):检查点是程序的一些敏感操作,如调用数据库查询语句,或是将数据返回到网页,如果这些操作的操作数据是污点,那么意味着操作可被攻击者利用,即程序存在漏洞。
	\item 清洁点(clean):清洁点通常是对污点进行消除的一类操作,如SQL注入、XSS中的过滤函数。清洁点是污点传播准确性的重要保证,不能识别清洁点即会引发污点过污染问题。
\end{itemize}


\subsubsection{污点分析过程}         
污点分析分为静态污点分析和动态污点分析,两者区别在于静态污点分析只使用程序代码模拟污点传播过程,而动态污点传播则通过程序的实际运行进行传播,由于本文关注于白盒测试情景,故只介绍静态污点传播方法,而在下一子章节会介绍动态传播的优劣势。

在定义好三要素之后,污点分析法会与数据流分析一样,对程序进行过程内分析和过程间分析。

过程内分析包括了显式流分析和隐式流分析,显示流分析即通过分析变量的数据依赖关系进行污点传播,而隐式分析则是指考虑控制依赖进行污点传播。

\begin{figure}[!htbp]
	\centering
	\includegraphics[width=0.5\linewidth]{FIGs/chapter2/internal_taintflow.png}
	\caption{过程内污点分析}\label{internalflow}
\end{figure}

如图~\ref{internalflow}所示,首先假设变量$a$为污点变量,实线箭头表示了显示污点传播路径,而虚线箭头表示了隐式污点传播路径,同时该图也说明了过程内污点传播基本思想,即从上至下遍历数据流图,若未标记的变量依赖于污点变量,则新变量也被标记为污点变量。虽然攻击者确实可以利用控制依赖操作数据进行攻击,但由于其分析复杂且会产生大量误报,在工程领域常常只做数据流依赖的显示分析,因此本文主要讨论显式流分析。

现代程序存在着复杂的函数调用,除了进行过程内分析,还需要进行过程间分析。其分析首先构造函数调用图(Call Graph),接着搜索存在产生点的函数,对于每一个存在产生点的函数,自顶向下分析(也可以自底向上分析)。遇到函数调用时,跟进被调函数,进行过程内污点分析,将分析结果表达为$\left\langle f, S, r\right\rangle$的函数摘要,其中$f$包含函数本身摘要信息(类名方法名和函数签名),$S$指调用过该函数后被污染的变量集合,$r$取值0或1,标记函数返回值是否被污染;接着根据函数摘要,再进行过程内分析,如此往复直至分析完函数所有代码块或是污点传播至汇聚点,报告漏洞。

\begin{figure}[!htbp]
	\centering
	\includegraphics[width=0.7\linewidth]{FIGs/chapter2/external_taintflow.png}
	\caption{过程间污点分析}\label{externalflow}
\end{figure}

如图~\ref{externalflow}所示,分析过程从左侧函数开始,因为其找到了一处产生点——\textit{request.getParameter("xss")},于是将污点传递到变量\textit{p},接着调用函数\textit{func(p)},于是对函数\textit{func}做过程内分析,得到其函数摘要,$\left\langle func, \left\{a, b, c\right\}, 1\right\rangle$,于是回到调用者的函数内,变量\textit{q}被标记为污点,又因为第三行存在一处汇聚点——\textit{response.getpriter.print()},并且参数为污点,于是报告此处有漏洞,并且根据汇聚点可以判断该漏洞是一个 XSS 漏洞。\\

\subsection{污点分析的优势和不足}
污点分析能够对程序上下文有一定理解,往往能产生误报率相对较低以及可解释的漏洞报告,其方法对 Web 类型的安全漏洞覆盖率较高,而污点类型的漏洞普遍具有较高的危害性~\cite{taintStyle,aletheia},因此该方法已被很多工业界、学术界的安全静态扫描工具所使用~\cite{taintStyle,taint:taj,pixy},本文也选择该技术产生初步的漏洞扫描结果。

然而,污点传播仍有可能发生误报,以下通过简单示例来说明。

\begin{figure}[!htbp]
	\centering
	\subfigure[污点传播难以处理容器类型]{
		\label{taintcase1}
		\includegraphics[width=0.4\textwidth]{FIGs/chapter2/tpcase1.png}}
	% \hspace{0.1in}
	\subfigure[污点传播难以分析控制流]{
		% \label{fig:subfig:b} %% label for second subfigure
		\label{taintcase2}
		\includegraphics[width=0.4\textwidth]{FIGs/chapter2/tpcase2.png}}
	\subfigure[污点传播难以处理特殊污染条件]{
		% \label{fig:subfig:b} %% label for second subfig ure
		\label{taintcase3}
		\includegraphics[width=0.4\textwidth]{FIGs/chapter2/tpcase3.png}}
	\subfigure[污点传播难以分析清洁函数]{
		% \label{fig:subfig:b} %% label for second subfigure
		\label{taintcase4}
		\includegraphics[width=0.4\textwidth]{FIGs/chapter2/tpcase4.png}}
	\caption{污点传播的不足}
	\label{fig:rq3} %% label for entire figure
\end{figure}
首先,污点传播对容器类型无法做很好处理,如图~\ref{taintcase1}所示,当污点传入容器类型时(在此例子中为\textit{map}),静态污点传播只能将这类变量的传播规则设为传播/不传播污点,从而造成过污染/欠污染,就如图所示,若设为\textit{map}传播污点,由于案例实际从容器中取出的是没有污点的变量,即过污染,而若设为不传播,若\textit{q}取出了参数\textit{p},那么又导致了欠污染。动态污点传播虽然解决了这一问题,但是由于其使用条件复杂,且无法用于静态分析,本文暂不讨论。

此外,静态污点传播对控制流没法做很好的处理,如图~\ref{taintcase2}所示,在第三行,程序已经对可能产生的 SSRF 漏洞进行了处理,即如果是内部地址的话则直接返回,但是不论是考虑显式流还是隐式流,污点传播都不能避免这一类误报。

再者,对于特殊触发条件的漏洞,污点传播无法很好处理,如图~\ref{taintcase3}所示,在第二行,因为 SSRF 要求攻击者能够操控主机名,所以即使用户输入的污点变量拼接在了一个正常网站之后,程序也不会出现 SSRF 的问题,而按照污点传播分析法,毫无疑问它会报告这段程序存在SSRF漏洞。

最后,不论是动态污点传播还是静态污点传播,其对污点清洁点的识别能力几乎为零,如图~\ref{taintcase4}所示,程序已经对潜在的XSS攻击做出的处理——即在第2$\sim$5行对用户输入的特殊符号进行替换和过滤,但是污点传播并不能识别这些清洁点,导致误报。    

正是因为存在这些不足,本文将在下文引入程序切片技术和 BLSTM 来降低误报率。\\

\subsection{Java污点分析工具选型}
Java上的污点分析工具有很多,如 Find Security Bugs,TAJ~\cite{taint:taj}。但对于安全扫描工具来说,除了污点传播引擎,针对漏洞定义好大量的入口点和汇聚点是扫描低漏报的重要保证。Find Security Bugs 作为较为经典的扫描工具,前人已经为预先定义了大量的污点传播入口点和汇聚点,并且支持自定义新的入口点和汇聚点,非常适合二次开发。

本系统保证准确性的手段在于利用机器学习对污点传播的误报进行改进,而不是定义大量污点传播规则,因此本文使用 Find Security Bugs 进行污点传播分析。此外,本系统解决了原版 Find Security Bugs 不能展示传播路径的问题,使其更加易用。\\

\section{程序切片技术}
\subsection{程序切片定义}
% https://www.cnblogs.com/maifengqiang/archive/2013/05/21/3090739.html
% https://hacpai.com/article/1555083057303
% 前向切片与后向切片之间关系的研究.pdf
程序切片技术是一种通过对程序分析,抽取程序中与关注点相关的一组语句集合的技术,目前广泛运用于程序调试,程序测试,优化,安全分析等领域。

该技术首次由Mark Weiser在其博士论文中提出~\cite{slices:weiser1979},他将程序切片做了如下定义:

将程序抽象为图$G\langle N,E\rangle$,$N$为程序中的语句集合,$E$为$\left\langle n,m \right\rangle$的集合,其中$n$为数据流的上一条语句,m为数据流的下一条语句。

\begin{definition}[程序状态序列(state trajectory)]
    
    若程序$P$中有长度为$k$的程序状态序列$T$,则:
    $$T=\left\langle \left\langle n_{1},s_{1} \right\rangle, \left\langle n_{2},s_{2} \right\rangle , \cdots , \left\langle n_{k},s_{k} \right\rangle \right\rangle$$
    其中$n_{i} \in N$,$s_i$为一个单射函数,记录所有变量到具体值的映射。
\end{definition}

\begin{definition}[切片准则(Slicing citerion)]
    
    若对于程序$P$有切片准则$C$,则$C=\left\langle i, V \right\rangle$,其中$i$指关注点,通常是指一条程序语句,$V$表示程序$P$中的变量子集(通常为$i$上的变量集合)。
    
\end{definition}

该切片准则决定了一个投影函数$Proj_C$:
$$
\operatorname{Proj}_{\langle i, V\rangle}(T)=\langle\operatorname{Proj}_{(i, V)}^{\prime}\left(t_{1}\right) , \cdots, \operatorname{Proj}_{\langle i, V\rangle}^{\prime}\left(t_{n}\right) \rangle
$$
其中:
$$
\operatorname{Proj}_{\langle i, V\rangle}^{\prime}(\left\langle n, s \right\rangle)=\begin{cases}
{\lambda} & {\text { if } n \neq i} \\
{\langle n, s | V \rangle} & {\text { if } n=i}
\end{cases}
$$


其中$\lambda$指空字符串,$s|V$指$V$中变量的单射函数。上式的含义是指,切片准则让我们只考虑$V$中变量的状态序列,并且只有当语句是关注点时,函数才返回状态序列,否则返回空字符。

\begin{definition}[程序切片]
    
    切片$S$为一组源程序语句的子集,它由不停地删除零条或多条原程序语句得到,同时保证$\operatorname{Proj}_{C}(T)=\operatorname{Proj}_{C^{\prime}}\left(T^{\prime}\right)$其中$T'$指切片中的状态序列,而$C'=\langle succ(i), V \rangle$,$succ(i)$指最靠近关注点$i$的语句(如果V不是$i$上的变量集合)或者就是$i$本身(如果$ V $是$i$上的变量集合)。
    
\end{definition}

按照定义,可以知道Mark Weiser实际上定义的是后向切片(backward slices),即切出对关注点造成影响的所有语句和谓词集合。对其改进后也出现了前向切片(forward slices),即切出被关注点影响的其他语句和谓词集合。\\

\subsection{程序切片技术}
%https://www.cnblogs.com/maifengqiang/archive/2013/05/21/3090739.html
程序切片技术自诞生以来,也经历了多个阶段~\cite{slices:xu2005brief},起初为Mark Weiser的概念阶段,主要通过控制流来进行程序切片;随后发展为基于依赖图的切片阶段,Ottenstein等人~\cite{slices:ottenstein1984program}首先提出了运用程序依赖图(PDG,Program Dependence Graph)进行切片,主要思想是首先对程序建立程序依赖图,再通过对程序依赖图的可达性计算进行后向切片,程序依赖图是一种反应语句数据依赖和控制依赖的一种程序中间表达形式,图中的点为单条程序语句,而图中的边则表示语句之间存在数据依赖关系,例如图~\ref{internalflow}的代码中,第三行与第二行就存在数据依赖关系,第七行与第四行则存在控制依赖关系。PDG只适用于单个函数的切片,horwitz等人~\cite{slices:horwitz1990}提出了系统依赖图(SDG, System Dependence Graph)的概念,在 PDG 的基础上,额外增加了两种边,一种边表示调用函数和被调用的直接依赖关系,另一种边描述由于边的依赖而引发的间接依赖关系,同时,前向切片也在此阶段产生;第三阶段为面向对象程序切片阶段,研究者提出将接受对象作为一个隐藏参数,结合多态解析手段,将面向对象切片问题转换为一般切片问题,使程序切片可以运用于面向对象语言。

目前已有很多工具可以对java程序进行切片,例如WALA(T.J. Watson Libraries for Analysis)~\footnote{\url{http://wala.sourceforge.net/wiki/index.php/Main_Page}}和Joana~\footnote{\url{https://pp.ipd.kit.edu/projects/joana/}}。WALA原本由IBM T.J. Watson 研究中心作为DOMO项目的一部分进行开发,在2006年,IBM将改软件转赠给社区。其主要功能包括Java类型系统和类层次分析,过程间数据流分析、指针分析和调用图构造以及上下文相关的切片等。Joana是由卡尔斯鲁厄理工学院推出,基于WALA 的一些分析手段构造的另一款 Java 切片工具,其包含指针分析,异常分析和PDG构图等复杂功能,并且其切片结果中能够更清晰的看到函数调用时实参的值,而对于污点传播类型中的自定义过滤函数,实参的值是一个很重要的特征(见图~\ref{fig:sliceresult}),因此本文选择Joana作为程序的切片器。\\
\begin{figure}[!htbp]
	\centering
	\subfigure[目标程序]{
		\includegraphics[width=0.42\textwidth]{FIGs/chapter2/slicetarget.png}}
	% \hspace{0.1in}
	\subfigure[Joana程序切片结果]{
		% \label{fig:subfig:b} %% label for second subfigure
		\includegraphics[width=0.42\textwidth]{FIGs/chapter2/joanaslice.png}}
	\subfigure[WALA程序切片结果]{
		% \label{fig:subfig:b} %% label for second subfigure
		\includegraphics[width=0.85\textwidth]{FIGs/chapter2/walaslice.png}}
	\caption{Joana和WALA程序切片示例}
	\label{fig:sliceresult} %% label for entire figure
\end{figure}

\subsection{后向程序切片的优势与不足}
程序切片能够得到影响关注点的程序语句,在有效缩短了程序上下文、降低被分析代码量的同时,还能有效保留控制流的信息,在程序调试,软件度量或是软件安全中已经得到了很广泛的应用。

由于污点传播已经对程序由上而下的进行了分析,通过对汇聚点和污点值进行后向程序切片,可以从反方向分析漏洞代码,暴露先前没有考虑的控制语句,同时通过进一步的特征处理,可以作为神经网络的输入,弥补污点传播对控制流分析不足的缺陷,因此本文采用该技术作为特征处理的基础工作,另外由于程序切片属于较为传统的技术,在java语言上已有较为成熟的工具,考虑到 Java 漏洞的特征与函数参数强相关这一特性,本文选择 Joana 进行程序切片。

然而,传统后向程序切片也存在不足,目前最大的挑战在于当程序规模变大时,后向程序切片效率低下;此外原生 Joana 并不支持在缺失程序依赖下对其程序进行切片,而在 SCA 的输入中,缺失依赖的程序又是非常常见的,针对以上问题本文做出了分段切片、限制调用图切片和无依赖切片的技术调整,具体细节将在第三四章节详细说明。\\

\section{BLSTM算法}
双向长短期记忆网络(BLSTM,Bidirectional Long Short-Term Memory)是一种特殊的循环神经网络(RNN,Recurrent neural network),主要改进有双向读取和增加记忆门两方面,是一种在自然语言处理上应用广泛的神经网络~\cite{lstm:translate}。\\

\subsection{LSTM原理介绍}
为了保证在不同的序列数据上,相同数据点有不同输出结果,人们将其记忆体加入神经网络之中,从而产生了RNN,然而RNN在处理较长文本时,存在着梯度爆炸或梯度消失的问题~\cite{lstm:gradient},Hochreiter和Schmidhuber于1997年在记忆体中加入写入、输出和遗忘门,而改进后的神经网络也被成为长短期记忆网络(LSTM,Long Short-Term Memory)~\cite{lstm:1997}。

\begin{figure}[htb]
	\centering
	\includegraphics[width=0.4\linewidth]{FIGs/chapter2/lstm_cell.pdf}
	\caption{LSTM的神经元}\label{lstmcell}
\end{figure}
如图~\ref{lstmcell}所示,一个LSTM的神经元有四个输入:$z$表示原输入,$z_i$表示输入门,$z_f$表示遗忘门,$z_o$表示输出门,以及有一个输出$o$,神经元内有一记忆体$C$,在下一时刻的输入进入神经元时,记忆体下一时刻的值$C'=g(z)*s(z_i)+C*f(z_f)$,$g(z)$为输入值的激活函数,$s(z)$为门限的激活函数,通常取sigmoid函数,从公式可以看出,首先输入门控制了输入数据本身是否可以输入进神经元,接着遗忘门控制了记忆体中旧数据是否会保留,最后输出$a=tanh(c')*f(z_o)$,显然,$z_o$控制了记忆体是否向外输出。

\begin{figure}[htb]
	\centering
	\includegraphics[width=0.8\linewidth]{FIGs/chapter2/lstm_time.pdf}
	\caption{LSTM的处理时序数据示意图}\label{lstmtime}
\end{figure}

利用LSTM处理时序数据情况如图~\ref{lstmtime}所示,对于任意时刻$t$,有$h_{t-1}$和$c_{t-1}$($t_0$时为随机值),$h_{t-1}$和$x_t$通过线性变换得到$z_f$、$z_i$、$z$和$z_o$,例如,$z_f=W_{f}\times [h,x]+b_{f}$。$z_f$、$z_i$、$z$和$z_o$均为以为向量,其长度等于该层LSTM神经元的个数,按上文所述,这4组向量输入神经元,结合神经元内的记忆体$c_{t-1}$计算得到输出$y_t$,同时$y_t$的值也作为下一轮输入的一部分,即$h_t$。$h_t$和$x_t+1$作为下一时刻的输入,输入下一时刻的神经元。在自然语言处理领域,人们通常利用最后一个时刻的输出对自然语言分类。\\

\subsection{双向读取——BLSTM}
早在RNN时期,就有研究者提出双向RNN提升拟合或预测准确性的思想,而LSTM作为RNN的一种,自然也可以运用该思想对其进行优化。

\begin{figure}[htb]
	\centering
	\includegraphics[width=0.8\linewidth]{FIGs/chapter2/blstm.pdf}
	\caption{BLSTM处理时序数据示意图}\label{blstm}
\end{figure}
如图~\ref{blstm}所示,对于一组时序数据$X=\left\{ x_0,\dots,x_t,x_{t+1},\dots,x_{k} \right\}$,训练两个LSTM网络,第一个网络正向读取数据(如图上方所示),第二个网络逆向读取数据(如图下方所示),在同一时刻,结合两个方向的网络输出值作为该时刻的输出(如果用于分类问题,通常结合正向$t_k$的输出和逆向$t_0$的输出)。


\subsection{BLSTM 的优势}

BLSTM的优势在于,单向LSTM在$x_t$时刻的输出实际上只被$x_0, x_1, \dots, x_t$影响,即其只能考虑截止到当前时刻的数据,而BLSTM的在某一时刻的输出值不仅受到该时刻正向数据的影响,还能受到该时刻后向数据的影响,因此有纵观全局的能力。\\

LSTM 增加了门限机制,解决了RNN 梯度爆炸或梯度消失的问题,而 BLSTM 通过双向传播机制,能够综合正向时序和逆向时序信息,是时序数据的分类和拟合问题(如自然语言处理问题)较为合适的算法。而目前已有很多利用自然语言处理和该模型进行程序分析的研究~\cite{naturalSoftware,lstm:recognize,lstm:repo,vuldeepecker,Koc2017,Koc2019}。所以本系统采用 BLSTM 作为污点传播预测算法,将污点传播的切片作为上下文敏感的自然语言数据,通过训练集训练 BLSTM 参数将其保存,在给定污点传播的切片输入时,系统调用训练好的模型对其进行预测。

\section{Django 框架}
\subsection{Django 框架简介}
Django 是一个高级的 Python Web 开发框架~\footnote{\url{https://www.djangoproject.com/}},它使开发者能够对 Web 应用进行简洁实用的设计并对其进行快速开发。Django 由一群有经验的开发者开发,其框架和内部设计避免了很多 Web 开发的问题,帮助开发者只关心于他们Web应用本身的逻辑而不需要重新开发通用组件。

Django 采用模型视图模板(MVT,Model-View-Template)模式,在模型层,其有一套高效的对象关系映射(ORM,Object Relational Mapping)框架和一些常用对象,比如用户对象,管理员对象等;在视图层,Django 阶级了基于正则表达式的URL分派系统,能够让用户灵活的处理自己的 URL 路由,并且内置的 Request,Response 对象能够让开发者快速的获取和返回数据;在模板层,Django 有自己的原生模板引擎,类似于 Jinja2,同时支持用户更换第三方引擎,保证用户充分自由的展现其数据。除此之外,Django 还内置了一个用户管理后台和各类安全保障模块,使大大加速了开发者的开发过程。\\

\subsection{Django 框架优势}
本系统的后台主要是对机器学习模型的操作,因此自然选择了 Python 开发语言,而 Django 作为 Python 语言上应用最为广泛的Web开发框架之一,自然也成为了本系统后台的首选技术。
利用 Django 的灵活性,本系统利用其 URL 分派系统接受客户端传来的预测、训练、标记等请求并对其处理,通过 ORM 框架将程序切片、模型训练数据快速本地化,以完成对系统后台的快速开发。

\section{本章小结}
本章首先介绍了目前流行的漏洞挖掘技术,并对其优劣势进行了说明,接着介绍了本系统使用的技术、工具和框架,并分别说明了使用这些技术、工具和框架的原因。在技术角度,本章依次介绍了污点分析技术、程序切片技术和 BLSTM 神经网络并对其优缺点进行了总结,分析了使用后向程序切片和 BLSTM 对污点传播会使污点分析更精确的原因;在工具角度,本章指出了使用 Find Security Bugs 进行污点分析,使用 Joana 进行程序切片的原因及本系统对其进行的优化改进;最后简单介绍了Django框架以及其优点。

\chapter{Java静态安全扫描工具需求分析与设计}
\section{系统整体概述}
Java语言通常用于Web服务和Android程序开发,其中的大量安全漏洞类型可以通过污点传播分析法进行扫描,然而随着漏洞种类和开发方式不断丰富,污点分析存在失误,因此产生了大量误报,为此安全工程师不得不修改污点传播规则,以增加漏报为代价换取低误报,或是人工对大量误报漏洞判断,这两种做法无疑会增加潜在安全风险。

本系统旨在提供对于Java代码的更精确的漏洞静态扫描功能,在源代码开发阶段进行代码安全扫描。相对于传统污点传播扫描工具,本系统结合了程序切片和BLSTM,从已有的误报漏洞片段进行分析辅助污点传播,从而降低误报率。系统总体架构如图~\ref{overview}所示,系统主要分为前端和后端两大部分,前端部分运行在用户端,包含污点分析模块、程序切片模块、部分预测模块,而在扫描系统后端运行在服务器端,包含预处理模块,另一部分预测模块和数据库。

污点分析模块对开源的find-sec-bugs做改进,使之能够输出污点传播树,为误报预测提供基础数据;程序切片模块分析污点传播树,将其拆分为子污点传播流,利用Joana生成SDG并进行切片;数据预处理模块将切片模块的输出数据进行泛化和向量化处理,使之可以输出预测模块;误报预测模块利用切片模块的子污点传播流和其对应切片,通过BLSTM进行切片预测,从而对漏洞进行预测。本系统采用客户端采用Java进行开发,管理后台和程序预测服务由Python和Django开发。

\begin{figure}[htb]
	\centering
	\includegraphics[width=5in]{FIGs/chapter3/system-architecture.pdf}
	\caption{系统总体架构图}\label{overview}
\end{figure}

\section{系统需求分析}
\subsection{功能性需求}
本系统功能性需求分析按污点分析模块,程序切片模块,预处理模块和误报预测模块四个方面进行。

污点分析模块由污点分析引擎对用户提交的jar文件进行扫描,产生初步结果,该模块所涉及的功能性需求如表~\ref{dmd:taint}所示,其主要功能为污点分析,而对于用户来说,系统需要友好的展现污点分析结果,即存在构造污点传播树和查看分析结果的需求,污点传播树由污点传播图构造而来,因此存在构造污点传播图需求,此外,污点分析是一项耗时操作,用户(尤其是开发工程师)可能会对一次污点分析结果进行保存,交给安全工程师进行漏洞验证,因此系统最好还需有保存分析结果和读取分析结果功能。
\begin{table}[!htbp]\footnotesize %label table
	\centering
	\caption{污点分析模块功能性需求列表}
	\vspace{2mm}
	% l - left, r - right, c - center. | means one vertical line 这里声明的是表格单元中的内容如何对齐
	\begin{tabular}{L{1.3cm}L{2.5cm}L{6.8cm}L{1.5cm}}
		\toprule
		\textbf{需求编号}&\textbf{需求名称}&\textbf{需求内容}&\textbf{优先级}\\
		\midrule
		R1	& 污点分析 				& 对于用户上传的代码进行污点分析,包括了过程内分析和过程间分析 & 高 \\
		R2  & 构造污点传播图 	 & 对于代码中存在的汇聚点,构造其污点传播图 & 高 \\
		R3  & 构造污点传播树	 & 对于一个污点传播类型的分析结果,用户最终看到一棵或多课污点传播树,其由污点传播图构造,一张污点传播图可能有多棵传播树,一个传播树上清晰标注有一个产生点和一或多个汇聚点 & 高 \\
		R4  & 查看分析结果	 & 对于一个污点传播类型的分析结果,用户最终看到一棵或多课污点传播树,对于树上的叶子节点,用户能够知晓其代码位置 & 高 \\
		R5  & 保存分析结果	   &用户可以对当前代码的漏洞分析详情进行保存,其中关键点在于对污点传播树进行序列化 & 中 \\
		R6  & 读取分析结果 	   &用户可以对保存后的分析结果文件进行读取,二次查看漏洞详情,其中关键点在于对污点传播树结构的反序列化 & 中 \\
		\bottomrule
	\end{tabular}
	\label{dmd:taint}
\end{table}

程序切片模块对污点分析结果进行切片处理,该模块所涉及的功能性需求如表~\ref{dmd:slice}所示,该模块的主要功能是根据污点分析模块的分析结果进行后向程序切片,将其拆解,得到四个子需求,首先其需要污点分析的结果进行处理,接着对污点传播树进行拆分,对于每一个污点传播流,其包含有若干个程序切片需求,下一步对这些需求进行后向切片,最后将切片结果转换为有一定结构的文本。

\begin{table}[!htbp]\footnotesize %label table
	\centering
	\caption{程序切片模块功能性需求列表}
	\vspace{2mm}
	% l - left, r - right, c - center. | means one vertical line 这里声明的是表格单元中的内容如何对齐
	\begin{tabular}{L{1.3cm}L{2.5cm}L{6.8cm}L{1.5cm}}
		\toprule
		\textbf{需求编号}&\textbf{需求名称}&\textbf{需求内容}&\textbf{优先级}\\
		\midrule
		R7	 & 处理污点分析结果 & 对污点分析的结果进行处理,对于每一个漏洞实例,反序列化的污点传播树并转化为适用于切片的传播树表达形式 & 高 \\
		R8	 & 污点传播树拆分 & 将每一个传播树拆分为一或多个污点传播流,每一个污点传播流包含若干个<函数入口点,关注点>对(其为一个切片的单位) & 高 \\
		R9   & 后向程序切片 & 对于每一个<函数入口点,关注点>, 结合用户上传的Jar包,进行后向程序切片 & 高 \\
		R10 & 输出切片结果	 & 对于切片,程序能将其表示为文本,即对切片结构体序列化 & 高 \\
		\bottomrule
	\end{tabular}
	\label{dmd:slice}
\end{table}

预处理模块对切片结果进行预处理,该模块所涉及的功能性需求如表~\ref{dmd:preprocessing}所示,该模块的主要功能是对切片文本进行泛化和向量化,在泛化需求主要保证一个切片能代表一类代码片段;随后进行向量化,对于切片的单词序列进行token标记,从而产生字典和切片向量,即BLSTM模型的输入。此外,一个较低优先级的需求为根据字典,将切片向量还原为切片的单词序列。

\begin{table}[!htbp]\footnotesize %label table
	\centering
	\caption{预处理模块功能性需求列表}
	\vspace{2mm}
	% l - left, r - right, c - center. | means one vertical line 这里声明的是表格单元中的内容如何对齐
	\begin{tabular}{L{1.3cm}L{2.5cm}L{6.8cm}L{1.5cm}}
		\toprule
		\textbf{需求编号}&\textbf{需求名称}&\textbf{需求内容}&\textbf{优先级}\\
		\midrule
		R11   & 泛化 & 对于每一个切片进行泛化处理,包括抽象数值、字符串、类名和方法名等 & 高 \\
		R12 & 向量化	 & 对于一个切片的单词序列,生成单词与整形值的字典并按字典对其向量化 & 高 \\
		R13 & 反向量化	 & 对于一个单词序列的向量,根据字典将其还原为切片的单词序列 & 低 \\
		\bottomrule
	\end{tabular}
	\label{dmd:preprocessing}
\end{table}

误报预测模块是对保障漏洞报告准确性的核心模块,包括客户端、后端和Web前端部分,该模块所涉及的功能性需求如表~\ref{dmd:predict}所示,该模块的需求主要为对给定漏洞实例进行预测,标记以及模型训练。预测依赖后端服务,因此有设置后端服务器的需求;预测分为预测切片安全性需求和预测漏洞真实性需求,分别由后端服务和客户端实现;标记分为对切片安全性标记和漏洞真实性标记,当漏洞标记为误报时,用户需指定安全的污染流;同时,管理员有训练预测模型的需求;此外,由于预测过程并不是很占用资源,对保存预测结果的需求程度较低;最后用户可以清空标记和预测结果。

\begin{table}[!htbp]\footnotesize %label table
	\centering
	\caption{误报预测模块功能性需求列表}
	\vspace{2mm}
	% l - left, r - right, c - center. | means one vertical line 这里声明的是表格单元中的内容如何对齐
	\begin{tabular}{L{1.3cm}L{2.5cm}L{6.8cm}L{1.5cm}}
		\toprule
		\textbf{需求编号}&\textbf{需求名称}&\textbf{需求内容}&\textbf{优先级}\\
		\midrule
		R14 & 预测切片安全性 & 对于一个切片向量,预测该切片是否是安全的(污染流无法传播) & 高 \\
		R15 & 标记切片安全性	 & 对切片向量,标记其是否安全 & 高 \\
		R16 & 预测漏洞真实性 & 用户在客户端可以查看预测结果,即该漏洞是否真实存在,需要结合切片预测需求 & 高 \\
		R17 & 标记漏洞真实性	 & 用户在客户端可以对于一个漏洞,标记漏洞是否存在,不存在时用户需提供理由(指出安全的污染流) & 高 \\
		R18 & 训练预测模型	 & 管理员可以在Web页面发起训练模型任务,或是安排定时器定期更新模型 & 高 \\
		R19 & 保存预测结果	 & 用户在客户端可以保存漏洞预测结果 & 低 \\
		R20 & 清空标记和预测结果	 & 用户在客户端可以清空标记和预测结果,以便重新预测 & 低 \\
		R21 & 设置预测服务器	 & 用户在客户端设置远程服务器 & 中 \\
		\bottomrule
	\end{tabular}
	\label{dmd:predict}
\end{table}


\subsection{非功能性需求}
本系统旨在为开发者和安全工程师提供Java代码安全扫描功能,考虑到系统功能性质、使用场景和所属领域,系统的非功能性需求主要有高效率,健壮性,保密性,安全性和扩展性五点。

首先,本系统的功能点在于对Java代码进行安全扫描,其使用场景位于软件开发至测试阶段,因此扫描过程要保证一定的效率和健壮性,如果一次扫描等待时间过长,或是扫描过程中途崩溃,那么开发过程就会受到影响,甚至影响整个服务的可用性,因此本系统需要是高效且健壮的。

其次,本系统涉及软件安全领域,很可能是恶意攻击者的首要攻击对象,因此自身需要具备保密性和安全性,这里的保密性是指保证用户提交的Java代码和Jar包的数据安全,安全性是指扫描服务本身经过充分安全性测试,不出现安全漏洞。

最后,很多企业目前已经拥有了适用于自身特点的基于污点传播的扫描器,为此,系统需要保证一定的扩展性,如将污点传播的解析抽象为接口,保证其可以基于任意污点传播工具进行预测。\\

\subsection{系统用例描述}
%用例图
本系统的主要用户为软件开发工程师和软件安全运营人员,根据对系统需求的分析,得到系统用例如图~\ref{fig:case}所示, 软件开发工程师和运营人员均有基于污点分析的静态扫描、扫描结果切片和预测、漏洞实例标记这三个用例,其中基于污点分析的静态扫描存在两例扩展用例,即保存污点分析结果和读取分析结果,漏洞实例标记实际包含了标记正报结果和标记误报结果用例,安全运营人员还多出模型训练和发放客户端口令的用例。

\begin{figure}[!htbp]
	\centering
	\includegraphics[width=5in]{FIGs/chapter3/case.pdf}
	\caption{系统用例图}\label{fig:case}
\end{figure}

\newcounter{caseCounter}
\setcounter{caseCounter}{1}

基于污点分析进行程序的静态扫描是本系统的基本功能,同样也是漏洞预测的基础。其用例描述如表~\ref{case:taint}所示,安全工程师或软件工程师(下面统称“用户”)对将开发好的项目进行构建,新建项目时将源代码和构建得到的Jar/War包添加到项目中,随后系统对其进行污点分析,并返回给用户潜在的安全漏洞。
% 1基于污点分析的静态扫描
\begin{table}[!htb]\footnotesize %label table
	\centering
	\caption{基于污点分析的静态扫描用例描述}
	\vspace{2mm}
	% l - left, r - right, c - center. | means one vertical line 这里声明的是表格单元中的内容如何对齐
	\begin{tabular}{L{3cm}L{10cm}}
		\toprule
		\textbf{描述项}&\textbf{说明}\\
		\midrule
		用例编号 & C\arabic{caseCounter}\stepcounter{caseCounter} \\
		用例名称 & 基于污点分析的静态扫描用例 \\
		对应功能需求编号  & R1, R2, R3, R4 \\ 
		参与者 & 用户 \\
		前置条件 & 用户已开发好项目并已构建的Jar/War包 \\
		后置条件 & 无\\
		正常流程 & \tabincell{l}{
							1. 用户选择“文件”,“新建项目”,打开新建项目会话框\\
							2. 用户输入项目名称,选择需要分析的包文件以及源文件所在目录\\
							3. 用户点击“分析”按钮,程序开始对其进行基于污点分析的静态扫描\\
							4. 用户在界面上看到漏洞列表,点击一个污点传播类型的漏洞\\
							5. 用户在界面上看到该漏洞的污点传播树\\
							6. 用户点击任意污点传播树的叶子节点,在界面上看到对应的源代码
					}\\
		异常流程 & 2. 若出现异常,则弹出异常会话框并显示异常信息\\
		\bottomrule
	\end{tabular}
	\label{case:taint}
\end{table}

% 2保存分析结果
污点分析后,用户可以对其分析结果进行保存,其用例描述见表~\ref{case:taintsave},用户点击保存按钮后,系统将污点传播树的每一节点序列化漏洞注解,并将含有一组注解的一系列漏洞保存为本地文件。

\begin{table}[!htb]\footnotesize %label table
	\centering
	\caption{保存污点分析结果用例描述}
	\vspace{2mm}
	% l - left, r - right, c - center. | means one vertical line 这里声明的是表格单元中的内容如何对齐
	\begin{tabular}{L{3cm}L{10cm}}
		\toprule
		\textbf{描述项}&\textbf{说明}\\
		\midrule
		用例编号 & C\arabic{caseCounter}\stepcounter{caseCounter}  \\
		用例名称 & 保存污点分析结果用例 \\
		对应功能需求编号  & R5 \\ 
		参与者 & 用户  \\
		前置条件 & 用户已创建项目且进行了基于污点分析的静态扫描 \\
		后置条件 & 用户指定位置出现保存好的分析报告\\
		正常流程 & \tabincell{l}{
			1. 用户选择“文件”,“另存为”,打开另存为对话框\\
			2. 用户输入保存文件名、文件类型和保存位置,点击保存按钮,分析结果\\被保存
		}\\
		异常流程 & 2. 若出现异常,则弹出异常会话框并显示异常信息\\
		\bottomrule
	\end{tabular}
	\label{case:taintsave}
\end{table}

% 3读取分析结果
对于以保存的分析结果,用户可以加载它们重新分析漏洞结果,程序对分析结果反序列化后在界面显示先前分析结果,如表~\ref{case:taintload}所示。

\begin{table}[!htb]\footnotesize %label table
	\centering
	\caption{读取分析结果用例描述}
	\vspace{2mm}
	% l - left, r - right, c - center. | means one vertical line 这里声明的是表格单元中的内容如何对齐
	\begin{tabular}{L{3cm}L{10cm}}
		\toprule
		\textbf{描述项}&\textbf{说明}\\
		\midrule
		用例编号 & C\arabic{caseCounter}\stepcounter{caseCounter}  \\
		用例名称 & 读取分析结果用例 \\
		对应功能需求编号  & R6 \\ 
		参与者 & 用户  \\
		前置条件 & 用户磁盘中已有一份污点分析报告 \\
		后置条件 & 无\\
		正常流程 & \tabincell{l}{
			1. 用户选择“文件”,“打开”,显示打开文件对话框\\
			2. 用户在打开文件对话框中选择报告文件,点击“打开”按钮,程序\\加载并在界面上显示污点分析报告中的内容
		}\\
		异常流程 & 2. 若出现异常,则弹出异常会话框并显示异常信息息\\
		\bottomrule
	\end{tabular}
	\label{case:taintload}
\end{table}

% 4设置预测服务器
对于客户端来说,对漏洞进行标记或者预测前提需要设置远程服务器,设置预测服务器的用例描述如表~\ref{case:setserver}所示,用户通过管理员发放的token和服务器地址设置远端服务器。
\begin{table}[!htb]\footnotesize %label table
	\centering
	\caption{设置预测服务器用例描述}
	\vspace{2mm}
	% l - left, r - right, c - center. | means one vertical line 这里声明的是表格单元中的内容如何对齐
	\begin{tabular}{L{3cm}L{10cm}}
		\toprule
		\textbf{描述项}&\textbf{说明}\\
		\midrule
		用例编号 & C\arabic{caseCounter}\stepcounter{caseCounter}  \\
		用例名称 & 设置预测服务器用例 \\
		对应功能需求编号  & R21 \\ 
		参与者 & 用户  \\
		前置条件 & 管理员已向用户发放token \\
		后置条件 & 无\\
		正常流程 & \tabincell{l}{
			1. 用户选择“AI”,“Set Server”,显示服务器设置对话框\\
			2. 用户服务器对话框汇中设置服务器地址以及token信息\\
			3. 用户点击“验证”按钮,检查配置是否正确\\
			4. 用户点击“应用”按钮,完成服务器配置
		}\\
		异常流程 & 2. 若出现异常,则弹出异常会话框并显示异常信息\\
		\bottomrule
	\end{tabular}
	\label{case:setserver}
\end{table}

% 5扫描结果预测
扫描结果切片和预测功能是本系统的最核心功能,程序首先对污点传播结果分析,拆解为污点传播片段并分别切片,再由服务端预测每一个切片是否为清洁片段,最后得出漏洞实例是否为真实漏洞。涉及的用例描述如表~\ref{case:predict}所示。

\begin{table}[!htbp]\footnotesize %label table
	\centering
	\caption{扫描结果切片和预测用例描述}
	\vspace{2mm}
	% l - left, r - right, c - center. | means one vertical line 这里声明的是表格单元中的内容如何对齐
	\begin{tabular}{L{3cm}L{10cm}}
		\toprule
		\textbf{描述项}&\textbf{说明}\\
		\midrule
		用例编号 & C\arabic{caseCounter}\stepcounter{caseCounter}  \\
		用例名称 & 扫描结果切片和预测用例 \\
		对应功能需求编号  & R7, R8, R9, R10, R14, R16 \\ 
		参与者 & 用户  \\
		前置条件 & 用户已完成远程服务器配置并且已有污点分析结果 \\
		后置条件 & 预测所需所有切片信息发送至服务器\\
		正常流程 & \tabincell{l}{
			1. 用户选择“AI”点击“Slice and Predict”,程序开始对污点分析结\\果进行切片和预测\\
			2. 用户等待切片和预测完成时,可以在弹出的会话框中观察预测进度,\\并随时取消预测\\
			3. 用户在程序主界面看到每一漏洞预测结果,预测为误报的漏洞由灰色\\图标标识且在漏洞列表标注为“(P: FP)”,预测结果为正报的漏洞图标\\不变,文字标注为“(P: TP)”\\
			4. 用户在点击一个漏洞,在污点传播图上可以看到预测解释,当预测结果\\为误报时,所有与预测为安全的污点传播流相关的叶子节点由“(*)”标\\注。
		}\\
		异常流程 & 2. 若出现异常,则弹出异常会话框并显示异常信息\\
		\bottomrule
	\end{tabular}
	\label{case:predict}
\end{table}

% 6标记正报结果
若漏洞实例被安全工程师判断为真实存在的,工程师可以对该漏洞标记为正报,此时所有污染流片段均自动标记为不安全,并且发送给服务端。标记正报结果用例描述如表~\ref{case:labeltp}所示。

\begin{table}[!htbp]\footnotesize %label table
	\centering
	\caption{标记正报结果用例描述}
	\vspace{2mm}
	% l - left, r - right, c - center. | means one vertical line 这里声明的是表格单元中的内容如何对齐
	\begin{tabular}{L{3cm}L{10cm}}
		\toprule
		\textbf{描述项}&\textbf{说明}\\
		\midrule
		用例编号 & C\arabic{caseCounter}\stepcounter{caseCounter}  \\
		用例名称 & 标记正报结果用例 \\
		对应功能需求编号  & R15, R17 \\ 
		参与者 & 用户  \\
		前置条件 & 用户已经对污点分析结果进行了切片和预测 \\
		后置条件 & 用户指定漏洞的所有污染流片段均标记为不安全且发送至服务器\\
		正常流程 & \tabincell{l}{
			1. 用户右键点击需要标记的正报漏洞实例,弹出标记菜单\\
			2. 用户点击“Labeled as True Positive”,该漏洞被标记为正报,漏洞说\\明内容被显示为“[L: TP]”
		}\\
		异常流程 & 2. 若出现异常,则弹出异常会话框并显示异常信息\\
		\bottomrule
	\end{tabular}
	\label{case:labeltp}
\end{table}

% 7标记误报结果
若漏洞实例被用户判断为不存在,用户可以对该漏洞标记为误报,此时用户需要指定一条最短的污染流,并且在该污染流中污点消失,同时,这段“安全的污染流”被发送至服务端。标记误报结果用例描述如表~\ref{case:labelfp}所示。

\begin{table}[!htb]\footnotesize %label table
	\centering
	\caption{标记误报结果用例描述}
	\vspace{2mm}
	% l - left, r - right, c - center. | means one vertical line 这里声明的是表格单元中的内容如何对齐
	\begin{tabular}{L{3cm}L{10cm}}
		\toprule
		\textbf{描述项}&\textbf{说明}\\
		\midrule
		用例编号 & C\arabic{caseCounter}\stepcounter{caseCounter}  \\
		用例名称 & 标记误报结果用例 \\
		对应功能需求编号  &  R15, R17 \\ 
		参与者 & 用户  \\
		前置条件 & 用户已经对污点分析结果进行了切片、预测或标记 \\
		后置条件 & 无\\
		正常流程 & \tabincell{l}{
			1. 用户右键点击需要标记的误报漏洞实例,弹出标记菜单\\
			2. 用户点击“Labeled Safe Flow”,弹出误报标记会话框\\
			3. 用户选择一条最短的安全污染流,同时看到该污染流的哈希和切片信\\息,点击确定后完成标记\\
			4. 用户看到其所标记的漏洞被标注为“[L: FP]”,并且在污染传播树中,\\所有与被用户标记的那段污染流相关的叶子节点由“[*]”标记。
		}\\
		异常流程 & 2. 若出现异常,则弹出异常会话框并显示异常信息\\
		\bottomrule
	\end{tabular}
	\label{case:labelfp}
\end{table}

% 8清空切片、标记和预测结果
由于切片、预测是复杂操作,系统会将其放入缓存,同时标记结果用户也会有想清空的情况,因此本系统提供一键清除切片、标记和预测结果的功能。其用例描述如表~\ref{case:clean}所示。
\begin{table}[!htb]\footnotesize %label table
	\centering
	\caption{清空切片、标记和预测结果用例描述}
	\vspace{2mm}
	% l - left, r - right, c - center. | means one vertical line 这里声明的是表格单元中的内容如何对齐
	\begin{tabular}{L{3cm}L{10cm}}
		\toprule
		\textbf{描述项}&\textbf{说明}\\
		\midrule
		用例编号 & C\arabic{caseCounter}\stepcounter{caseCounter}  \\
		用例名称 & 清空切片、标记和预测结果用例 \\
		对应功能需求编号  & R20 \\ 
		参与者 & 用户  \\
				前置条件 & 用户已经对污点分析结果进行了切片和预测 \\
		后置条件 & 用户指定漏洞的最短安全污染流被标记为安全且发送至服务器\\
		正常流程 & \tabincell{l}{
			1. 用户选择“AI”点击“Clean DB”,程序清空切片、预测和标记数据,\\并在漏洞的预测标签上显示“(P: UNK)”,在标记标签上显示“[L: UNK]”
		}\\
		异常流程 & 无\\
		\bottomrule
	\end{tabular}
	\label{case:clean}
\end{table}

% 9模型训练
模型训练的用例描述如表~\ref{case:train}所示,管理员通过在Web后端发起模型训练请求,系统会进行异步调用,通过消息队列,训练程序会根据选择的训练配置进行模型训练,并将训练模型保存,供预测时调用,用户也可以设置定时任务,是系统按时自动训练模型。
\begin{table}[!htb]\footnotesize %label table
	\centering
	\caption{模型训练用例描述}
	\vspace{2mm}
	% l - left, r - right, c - center. | means one vertical line 这里声明的是表格单元中的内容如何对齐
	\begin{tabular}{L{3cm}L{10cm}}
		\toprule
		\textbf{描述项}&\textbf{说明}\\
		\midrule
		用例编号 & C\arabic{caseCounter}\stepcounter{caseCounter}  \\
		用例名称 & 模型训练用例 \\
		对应功能需求编号  & R11, R12, R18 \\ 
		参与者 & 用户(这里特指安全运营人员)  \\
		前置条件 & 数据库中已存在用于训练的切片和标记数据 \\
		后置条件 & 数据库中记录已经训练好的模型\\
		正常流程 & \tabincell{l}{
			1. 用户登录 Web 控制台\\
			2. 用户在控制台中点击“Model Config”进入模型配置页面,查看或新\\建模型配置\\
			3. 用户在控制台中点击“Periodic Tasks”,进入任务页面,在任务页面\\
			中选择模型训练任务,并设置任务类型(定时任务或单次执行任务)并\\且指定时间,以及训练指定的模型配置,系统到规定时间后自动训练模\\型
		}\\
		异常流程 & \tabincell{l}{4. 若模型训练中遇到错误,将错误信息返回到任务结果表中,\\供用户查看}\\
		\bottomrule
	\end{tabular}
	\label{case:train}
\end{table}

客户端令牌管理的用例描述如表~\ref{case:train}所示,管理员通过在Web后端对客户端令牌进行管理,即对其进行增删改查操作。
\begin{table}[!htb]\footnotesize %label table
	\centering
	\caption{客户端令牌管理用例描述}
	\vspace{2mm}
	% l - left, r - right, c - center. | means one vertical line 这里声明的是表格单元中的内容如何对齐
	\begin{tabular}{L{3cm}L{10cm}}
		\toprule
		\textbf{描述项}&\textbf{说明}\\
		\midrule
		用例编号 & C\arabic{caseCounter}\stepcounter{caseCounter}  \\
		用例名称 & 客户端令牌管理用例 \\
		对应功能需求编号  & R21 \\ 
		参与者 & 安全运营人员  \\
		前置条件 & 无 \\
		后置条件 & 无 \\
		正常流程 & \tabincell{l}{
			1. 用户登录 Web 控制台\\
			2. 用户在控制台中点击“Client Token”进入客户端令牌管理页面,对客\\户端令牌进行增加,修改,删除等操作。
		}\\
		异常流程 & 无 \\
		\bottomrule
	\end{tabular}
	\label{case:token}
\end{table}

\section{系统总体设计}
本系统的框架图已在本节第一章图~\ref{overview}展示,下面将对系统设计的模块从逻辑视图、开发视图、进程视图、物理视图和场景视图五个方面详细介绍系统总体设计。% The 4+1 View Model of Architecture,Philippe Kruchten 

首先是逻辑视图,该视图主要反应本系统的对象设计,如图~\ref{logicview}所示,图中 AbstractTaintDetector 是污点传播分析类的抽象类,与之相关的类有MethodDescriptor 和 Location ,其分别表示函数摘要和代码位置;
AbstractIncjectionDetector 是注入类型漏洞分析器的抽象类,继承自 AbstractTaintDetector,相对于 AbstractTaintDetector 其主要增加了具有输出漏洞报告的功能,目前几乎所有污点传播漏洞类都继承该类;
与 AbstractIncjectionDetector 相关的有 InjectionSink 类,该类表示污点汇聚点,其记录汇聚点可产生污点传播图,即 TaintFlowGraph 类;
与 TaintFlowGraph 相关的有 TaintTreeGenerator ,即污点传播树生成类,其用于生成污点传播树,树上节点为 TaintTreeNode 类;
BugAnnotationWithSourceLines 是漏洞注解类,用于在用户界面上显示一个污点传播树的叶子节点;
若干棵污点传播树共同构成了一个漏洞实例——BugInstance类;若干个BugInstance类组合产生BugCollection,即一个项目的漏洞集合类;
SliceRunner 为切片控制类,其首先调用报告翻译器对漏洞报告进行翻译,产生污点项目类——TaintProject,接着调用切片器对程序进行切片,最终产生切片项目类——SliceProject;ReportParser 和 Slicer分别为翻译器和切片器的接口;
SpotbugsReportParser 为 Spotbugs 的报告翻译器,实现了翻译器接口,负责筛选find-sec-bugs的污点传播类型漏洞并进行翻译;
JoanaSlicer 为基于 Joana 的切片器,实现切片器接口;
切片项目中包含切片的集合,切片类为 Slice;
PredictRunner 为预测控制类其依赖预测器接口——Predictor 对漏洞实例进行预测,返回为预测项目类——PredictProject 的实例;
BLSTMRemotePredictor 为远程调用服务端预测接口的预测器,实现 Predictor 接口;
后端服务器的所有服务表示为 Service 类,其中包含对 BLSTM 模型的控制类 ModelController;
表示预处理的类有 Preprocessing 类,该类中有对切片的各类泛化操作,以及 Tokenizer 类,其中包含对切片的向量化和反向量化操作;
最后模型类表示为 BLSTM 类;其通过数据集 Dataset 类进行训练。

\begin{figure}[!htb]
	\centering
	\includegraphics[width=5in]{FIGs/chapter3/logicview.pdf}
	\caption{系统逻辑视图}\label{logicview}
\end{figure}

% 逻辑视图=UML类图
% 开发视图=开发者角度,模块框图,加切片、预测模型
% 进程视图?
% 部署图,组件图

% 场景视图=用例图

\section{污点分析模块设计}
\subsection{类图设计}
函数调用图数据结构:

污染传播树数据结构:

\subsection{流程设计}

\section{程序切片模块设计}
\subsection{类图设计}
\subsection{流程设计}

\section{数据预处理模块设计}
\subsection{类图设计}
\subsection{流程设计}

\section{误报预测模块设计}
\subsection{架构设计}
\subsection{类图设计}
\subsection{流程设计}

\section{数据库设计}
本系统的数据库主要用来存储用户身份信息、污点分析得到的污染流数据和标记、用于模型训练的参数配置信息以及模型本身信息。经过分类,可以总结为三类数据表:

第一类是与鉴权有关的数据表,由于本系统主要使用了C/S架构,为了防止恶意攻击者向服务器注入垃圾数据影响预测,因此需要对客户端身份进行鉴定,为此设立客户端口令(Client\_Token)表,由安全运营人员进行维护。

第二类是与程序分析相关的数据表,由于本系统后台主要是对子污染流的切片进行有监督预测,因此自然需要污染流(Taint\_Flow)表和对污染流的标记(Label)表,对一个污染流而言,需要对其函数入口和关注点(即语句位置)进行切片,因此存在函数摘要(Method\_Description)表和语句位置(Location)表,不同项目可能具有名称相同的类、函数或文件,为了在逻辑上唯一标识一个函数摘要和语句位置,因此建立项目(Project)表。

第三类是与模型本身相关的数据表,本系统使用BLSTM进行切片预测,对BLSTM训练和BLSTM结构本身涉及到一系列参数,为此将其保存在模型配置(Model\_Config)表中,供安全运营人员调整,对于已经训练好的模型,将其保存在BLSTM模型(LSTM\_Model)表中。

与鉴权有关的数据表实体关系图如图~\ref{er:token} 所示,其中只有一张表,即客户端口令表,其中字段如表~\ref{sql:tokenTable} 所示,id 为一个口令的唯一标识;token 为一字符串,由管理员设置和发放,客户端连接时需要有一合法 token 才可与服务器通信;description 用于保存该口令的一些描述信息;create\_time 为该口令的创建时间。
\begin{figure}[!htbp]
	\centering
	\includegraphics[width=0.5\linewidth]{FIGs/chapter3/token_er.pdf}
	\caption{与鉴权相关表的实体关系图}\label{er:token}
\end{figure}

\begin{table}[!htbp]\footnotesize %token table
	\centering
	\caption{Client\_Token 表}
	\vspace{2mm}
	% l - left, r - right, c - center. | means one vertical line 这里声明的是表格单元中的内容如何对齐
	\begin{tabular}{L{2cm}L{2cm}L{2.6cm}L{6cm}}
		\toprule
		\textbf{字段名}&\textbf{数据类型}&\textbf{属性}&\textbf{说明}\\
		\midrule
		id					&INT&PK&主键,token的唯一标识\\
		token 				&VARCHAR(255)&NN, UNIQUE&表示管理员向用户发放的 token\\
		description				 &VARCHAR(255)& N &用于保存管理员对此 token 的备注信息\\
		create\_time		  &DATETIME&NN&表示 token 的创建时间\\
		\bottomrule
	\end{tabular}
	\label{sql:tokenTable}
\end{table}

与程序分析相关的实体关系图如图~\ref{er:program},其中,项目表保存项目名称,函数摘要表保存一个项目中的函数摘要信息,语句位置表记录关注点所在的代码位置,污染流表记录污染流和其中的切片信息,标记表记录一个污染流记录是否是安全的。一个项目可以有多个函数摘要、位置和污染流,即有一对多的关系,一个函数摘要或语句位置可以有多个污染流记录,而标记和污染流记录呈一对一关系。

\begin{figure}[!htbp]
	\centering
	\includegraphics[width=1\linewidth]{FIGs/chapter3/program_er.pdf}
	\caption{与程序分析相关表的实体关系图}\label{er:program}
\end{figure}

表~\ref{sql:project}是对项目表的字段说明,其中,id是一个项目的唯一标识;name标识项目名称,存在唯一性约束,由客户端建立扫描项目时指定;create\_time记录该项目的创建时间。

\begin{table}[!htbp]\footnotesize %project table
	\centering
	\caption{Project 表}
	\vspace{2mm}
	% l - left, r - right, c - center. | means one vertical line 这里声明的是表格单元中的内容如何对齐
	\begin{tabular}{L{2cm}L{2cm}L{2.6cm}L{6cm}}
		\toprule
		\textbf{字段名}&\textbf{数据类型}&\textbf{属性}&\textbf{说明}\\
		\midrule
		id							&INT&PK&主键,项目的唯一标识\\
		name		 			&VARCHAR(255)&NN&表示项目的名称\\
		create\_time		  &DATETIME&NN&表示项目创建时间\\
		\bottomrule
	\end{tabular}
	\label{sql:project}
\end{table}

表~\ref{sql:methodDescription}是对函数摘要表的字段说明,其中,id 是一个函数摘要的唯一标识;clazz 记录函数的类名;method 记录函数的方法名;sig 记录函数签名,其包括了参数类型和返回值类型;project 作为外键,标识该函数摘要位于哪个项目中;create\_time记录该函数摘要的创建时间。此外,clazz,method,sig,project能够唯一标识一个函数摘要。
\begin{table}[!htbp]\footnotesize %methodDescription table
	\centering
	\caption{Method\_Description 表}
	\vspace{2mm}
	% l - left, r - right, c - center. | means one vertical line 这里声明的是表格单元中的内容如何对齐
	\begin{tabular}{L{2cm}L{2cm}L{2.6cm}L{6cm}}
		\toprule
		\textbf{字段名}&\textbf{数据类型}&\textbf{属性}&\textbf{说明}\\
		\midrule
		id					&INT&PK&主键,函数摘要的唯一标识\\
		clazz				&VARCHAR(255)&NN&表示该函数的类名\\
		method 			& VARCHAR(255) &NN&表示该函数的方法名\\
		sig					&VARCHAR(255)&NN&表示该函数的签名,即函数参数和返回值类型\\
		project  		  &INT&FK, NN&外键,指向Project,表示该函数出现在此项目中\\
		create\_time  &DATETIME&NN&表示标记创建时间\\
		\bottomrule
	\end{tabular}
	\label{sql:methodDescription}
\end{table}

表~\ref{sql:location}是对语句位置表的字段说明,其中,id是一个语句位置的唯一标识;source\_file记录该语句所在的文件名	;start\_line记录该位置开始的代码行号;end\_line记录该位置结束的代码行号  ;project作为外键,记录该位置位于哪个项目中;create\_time记录该项目的创建时间。此外,source\_file,start\_line,end\_line,project联合可以唯一标识一个语句位置。
\begin{table}[!htbp]\footnotesize %location table
	\centering
	\caption{Location 表}
	\vspace{2mm}
	% l - left, r - right, c - center. | means one vertical line 这里声明的是表格单元中的内容如何对齐
	\begin{tabular}{L{2cm}L{2cm}L{2.6cm}L{6cm}}
		\toprule
		\textbf{字段名}&\textbf{数据类型}&\textbf{属性}&\textbf{说明}\\
		\midrule
		id							&INT&PK&主键,语句位置的唯一标识\\
		source\_file		 	&VARCHAR(255)&NN&表示语句所在文件名\\
		start\_line 			& INT &NN, UNSIGNED&表示语句所在的开始行号\\
		end\_line				&INT&NN, UNSIGNED&表示语句所在的结束行号\\
		project  			  &INT&FK, NN&外键,指向Project,表示该位置出现在此项目中\\
		create\_time		  &DATETIME&NN&表示标记创建时间\\
		\bottomrule
	\end{tabular}
	\label{sql:location}
\end{table}

表~\ref{sql:taintFlow} 是对污染流表的字段说明,其中,id 是一个污染流的唯一标识;hash	是这个污染流的哈希,通过切片文本的SHA1得到;slice\_file 切片文件在硬盘上的保存位置,当数据库中记录删除时,切片文件也应被删除;entry和point分别为切片的入口点和关注点 ;project作为外键,记录污染流存在于哪个项目中;create\_time记录该污染流的创建时间。此外,entry,point和project可以联合唯一标识一个污染流记录。

\begin{table}[!htbp]\footnotesize %taintFlow table
	\centering
	\caption{Taint\_Flow 表}
	\vspace{2mm}
	% l - left, r - right, c - center. | means one vertical line 这里声明的是表格单元中的内容如何对齐
	\begin{tabular}{L{2cm}L{2cm}L{2.6cm}L{6cm}}
		\toprule
		\textbf{字段名}&\textbf{数据类型}&\textbf{属性}&\textbf{说明}\\
		\midrule
		id					&INT&PK&主键,污染流的唯一标识\\
		hash				&VARCHAR(255)&NN&表示污染流的哈希\\
	    slice\_file			& VARCHAR(255) &NN&表示污染流切片的文件名\\
		entry				&INT&FK&外键,指向Method\_Description,表示污染流切片的入口\\
		point				&INT&FK&外键,指向Location,表示污染流切片的结束位置\\
		project  		  &INT&FK, NN&外键,指向Project,表示污染流所在的项目\\
		create\_time  &DATETIME&NN&表示污染流的创建时间\\
		\bottomrule
	\end{tabular}
	\label{sql:taintFlow}
\end{table}

表~\ref{sql:label}是对标记表的字段说明,其中,id 是一个标记的唯一标识;taint\_flow 作为外键,指这被标记的污染流对象,同时一个污染流同时只能有一个标记;entry和point分别为切片的入口点和关注点 ;is\_safe 标记了该污染流是安全/不安全的;create\_time记录该标记的创建时间;update\_time记录该标记的更新时间。

\begin{table}[!htbp]\footnotesize %label table
	\centering
	\caption{Label 表}
	\vspace{2mm}
	% l - left, r - right, c - center. | means one vertical line 这里声明的是表格单元中的内容如何对齐
	\begin{tabular}{L{2cm}L{2cm}L{2.6cm}L{6cm}}
		\toprule
		\textbf{字段名}&\textbf{数据类型}&\textbf{属性}&\textbf{说明}\\
		\midrule
		id							&INT&PK&主键,标记的唯一标识\\
		taint\_flow		 		&INT &FK, NN&外键,指向Taint\_Flow,表示本标记的污染流对象\\
		is\_safe 				& BOOLEAN &NN&表示这个污染流是安全的\\
		create\_time		  &DATETIME&NN&表示标记创建时间\\
		update\_time		&DATETIME&NN&表示标记更新时间\\
		\bottomrule
	\end{tabular}
	\label{sql:label}
\end{table}

与模型相关的实体关系图如图~\ref{er:model} 所示,其中,模型配置(Model\_Config)表记录模型配置信息,主要包括了配置名,模型训练相关信息和 LSTM 模型的参数等,LSTM 模型表(LSTM\_Model)记录以训练好的模型信息,主要信息有训练度量,模型保存位置等,一个配置可以训练多个模型。

\begin{figure}[!htbp]
	\centering
	\includegraphics[width=1\linewidth]{FIGs/chapter3/model_er.pdf}
	\caption{与模型相关表的实体关系图}\label{er:model}
\end{figure}

 表~\ref{sql:modelConfigTable} 是对模型配置表的字段说明,其中 name 作为主键,用于唯一标识一个配置,训练时通过指定改值进行训练;slice\_dir 和 label\_dir 记录了学习用的数据集位置,分别表示了切片所在文件夹和标记所在文件夹(模型直接通过文件进行读取,而不从数据库读取,减少了IO消耗);embedding\_dim,hidden\_dim 为模型自身参数,分别代表词嵌入的维度和隐藏层神经元个数(模型的每层神经元个数相等);word\_freq\_gt 表示了将出现次数小于该值的单词表示为“UNK”(Unknown);early\_stop\_patience,base\_learning\_rate,batch\_size,epoch,train\_percent记录了模型训练的配置,early\_stop\_patience 值训练轮数超过改值后,若模型效果(loss)仍比之前最优效果差,则提前停止训练,base\_learning\_rate 指初始学习率,batch\_size 指批量传入模型的数据量,epoch 指最大训练轮数,train\_percent 指训练集占总数据集的百分比,剩下的数据作为训练集,用于评估训练效果;最后,create\_time 和 update\_time 分别表示配置创建时间和配置更新时间。

\begin{table}[!htbp]\footnotesize %model config table
	\centering
	\caption{Model\_Config 表}
	\vspace{2mm}
	% l - left, r - right, c - center. | means one vertical line 这里声明的是表格单元中的内容如何对齐
	\begin{tabular}{L{2.2cm}L{2cm}L{2.6cm}L{6cm}}
		\toprule
		\textbf{字段名}&\textbf{数据类型}&\textbf{属性}&\textbf{说明}\\
		\midrule
		name 						&VARCHAR(255)&PK&主键,唯一标识一个配置\\
		slice\_dir		 			&VARCHAR(255)&NN&模型使用切片数据所在的文件夹\\
		label\_dir		 			&VARCHAR(255)&NN&模型使用标记数据所在的文件夹\\
		embedding\_dim		  &INT&NN, UNSIGNED&词嵌入时的维度,值为正整数\\
		hidden\_dim				&INT&NN, UNSIGNED&隐藏层神经元的个数,值为正整数\\
		word\_freq\_gt			&INT&NN, UNSIGNED&最小词频,小于该词频则表示为UNK,值为正整数\\
		early\_stop\_patience		&INT&NN, UNSIGNED&提前停止忍耐度,如果学习轮数大于此值,效果仍比以前差则学习停止,值为正整数\\
		base\_learning\_rate		&DOUBLE&NN, UNSIGNED&初始学习率\\
		batch\_size					&INT&NN, UNSIGNED&批量数据记录数,每次向神经网络输入batch\_size条记录,再进行反向传播\\
		epoch						&INT&NN, UNSIGNED&训练最大迭代次数,超过该次数则停止训练\\
		train\_percent						&DOUBLE&NN, UNSIGNED&表示训练集占整个数据的百分比,默认为1,即将所有数据代入训练\\
		create\_time				&INT&NN, UNSIGNED&配置创建时间\\
		update\_time				&INT&NN, UNSIGNED&配置更新时间\\
		\bottomrule
	\end{tabular}
	\label{sql:modelConfigTable}
\end{table}

表~\ref{sql:lstmModelTable} 是对 BLSTM 模型表的字段说明,其中 id 作为主键唯一标识一个模型;token\_dict\_path 和 model\_path 分别表示单词到整形值的映射字典文件位置和BLSTM模型位置,这两个字段用于加载模型实现对切片的预测;config 表示模型是由何配置训练得到的;accuracy、precision、recall和F1表示了模型的四种度量,当配置划分没有训练集时,这些值为-1;create\_time记录模型被训练出的时间。

\begin{table}[!htbp]\footnotesize %lstm model table
	\centering
	\caption{BLSTM\_Model表}
	\vspace{2mm}
	% l - left, r - right, c - center. | means one vertical line 这里声明的是表格单元中的内容如何对齐
	\begin{tabular}{llll}
		\toprule
		\textbf{字段名}&\textbf{数据类型}&\textbf{属性}&\textbf{说明}\\
		\midrule
		id&INTEGER&PK&主键,模型的唯一标识\\
		token\_dict\_path &VARCHAR(255)&NN&保存单词到tokenid的字典文件位置\\
		model\_path		 &VARCHAR(255)&NN&保存BLSTM模型文件的位置\\
		config				 &VARCHAR(255)&FK,NN&指向Model\_Config表,表示模型对应的配置信息\\
		accuracy	&DOUBLE&NN&表示学习后训练集的准确率,若无训练集则为-1\\
		precision	&DOUBLE&NN&表示学习后训练集的精确率,若无训练集则为-1\\
		recall	&DOUBLE&NN&表示学习后训练集的召回率,若无训练集则为-1\\
		F1	&DOUBLE&NN&表示学习后训练集的F1,若无训练集则为-1\\
		create\_time		  &TIME&NN&记录模型创建时间\\
		\bottomrule
	\end{tabular}
	\label{sql:lstmModelTable}
\end{table}

\section{本章小结}

\chapter{Java静态安全扫描工具实现和测试}
\section{污点分析模块的实现}
\subsection{构造污点传播函数调用图}
%数据结构
%构造方法
\subsection{构造污点传播树}
%数据结构
%构造方法
%数据保存形式
\subsection{关键代码}

\section{程序切片模块的实现}

\subsection{后向切片接口和实现}
%突出解决了的几个坑(bug,无依赖切片,限制调用图的切片)

\subsection{切片控制模块实现}
%突出流程图

\subsection{关键代码}

\section{数据处理模块的实现}
目前还未实现,先不写\\
\subsection{关键代码}

\section{误报预测模块的实现}

\subsection{误报预测控制模块实现}
%突出流程图

\subsection{关键代码}

\section{UI模块的实现}
目前还未实现,先不写\\
\subsection{关键代码}

\section{系统测试与运行展示}
\subsection{测试目标}
\subsection{功能测试}
\subsection{性能测试}
\subsection{系统运行展示}
\subsection{系统效果评估}

\chapter{总结与展望}
\section{总结}
为了解决应用安全问题,本文设计了一套基于污点分析、程序切片和 BLSTM 的静态安全扫描系统,旨在从源头出遏制程序漏洞。相较于传统的污点分析类代码扫描系统,本系统对污点传播树进一步分析,通过程序切片和先前标记进行学习,能够有效地排除误报,保证扫描结果准确性,大大降低了系统使用时的人力成本。本文主要工作如下:


\begin{enumerate}
    \item 本文对开源污点传播工具 FindSecBugs 进行改造,使之能在报告中展示污点传播树,友好地向用户展示可能的漏洞利用过程。
    \item 系统对污点传播树进行拆分,将污点传播流和子传播流作为切片单位进行程序切片,系统优化了 Joana 切片过程,使之能够在缺失依赖、异构 Jar 包的情况下切片,并通过限制调用图限制程序切片范围,结合污点传播树的拆分,解决了目前切片资源消耗大导致扫描任务无法及时结束的问题。
    \item 系统根据实际情况对切片后的 SSA 进行泛化处理,进一步保证预测模型的泛化能力。
    \item 系统通过对漏洞的各个污点传播树的传播流及其子传播流进行预测,从而预测漏洞是否为误报。系统采用 BLSTM 模型,该模型在学术界已被多项工作证明其具在漏洞预测领域有较为显著的效果。用户在系统界面中能够得知漏洞预测结果,若为误报,系统能够清晰给出判断依据,即污点在哪一段传播流中无法继续传播,一定程度上弥补了深度学习可解释性差的问题。安全工程师能够轻松地对漏洞、传播树和传播流进行标记,并发起新的学习任务,不断提高预测模型的准确率。
\end{enumerate}

系统被设计为有良好的可拓展性,对于污点分析结果翻译、切片器和预测器,系统都对其高度抽象为接口,方便后期开发者在后期根据实际需要对系统中各个模块所用技术进行更换。

经过测试和实验,结果表明本系统具有较高的健壮性,在 Maven 仓库中前 100 流行度的项目中系统扫描成功率为 100 \%,此外,本系统具有较高准确性,在 OWASP Benchmark v1.1 和 Juliet Test Suite for Java v1.3 的混合数据集中,本系统报告的准确性远高于传统污点分析类工具,相较于 FindSecBugs,本系统提高了 xx \% 的准确率,减少了 xx\% 的误报。

\section{展望}

本系统是将深度学习在应用于漏洞扫描领域的一次成功探索,实现了传统检测技术与深度学习相结合,对于 Java 代码进行更准确的安全扫描任务,然而在未来,系统在以下方面还有较大的改进空间:

\begin{enumerate}
    \item 系统目前只针对 Java 语言,在未来可以将其方法推广到其他语言的项目中。
    \item 系统只能针对部分误报模式的消除,在未来可以针对异步调用、反射调用等其他误报模式进行处理,进一步消除误报。
    \item 系统并不能发现更多漏报,在未来可以使用类似的切片技术和预测方法,结合其他经典漏洞挖掘技术,在消除漏报方面做进一步提升。
    \item 系统在特征表示时,目前实际上是将切片转化为单词序列,再将单词序列进行向量化处理,在未来可以参考程序图特征表示的前沿工作,将切片表示成信息更丰富的特征,进一步提高预测准确性。
\end{enumerate}

% 参考文献

\bibliography{reference}
%% addde by lhy
%% 因为overleaf上没有这个参考文献排版文件
% \bibstyle{elsart-num}

%个人简介
\Nchapter{简历与科研成果}
\noindent {\heiti 基本情况}
\vspace{1ex}
\noindent 徐文远,男,汉族,1995~年~8~月出生,江苏省南京市人。
\vspace{2ex}

\noindent {\heiti 教育背景}
\begin{description}[labelindent=0em, leftmargin=8em, style=sameline]
\item[2018.9~2020.6] 南京大学软件学院 \hfill 硕士
\item[2014.9~2018.6] 扬州大学信息工程学院 \hfill 本科
\end{description}
% 发表文章目录

\noindent {\heiti 这里是读研期间的成果(实例为受理的专利)}

\begin{enumerate}[label=\arabic*., labelindent=0em, leftmargin=*]
	\item 李四,\textbf{张三},``一种使用Hammer砸碎Nut的方法'',申请号:20xx1018xywz.a,已受理。
\end{enumerate}


\backmatter


\begin{thanks}

\vskip 18pt

这里是致谢。
一般的感谢顺序:导师,其他指导老师,师兄弟姐妹、同学,父母和伴侣。

\end{thanks}


\end{document}


